\section{Introdução}

Durante o desenvolvimento dos computadores digitais o maior problema encontrado
é o tempo de processamento. Com a criação de novas necessidades para a raça humanas
sejam elas essenciais ou apenas para o lazer, os computadores se tornaram 
itens obrigatório, e quanto mais rápido ele são, mais tarefas podem desempenhar.
Um operação que requer um quantidade de processamento considerável é a multiplicação,
uma operação que surge naturalmente em qualquer modelo matemático. Diversos
métodos digitais foram desenvolvidos para está operação. Um dos mais conhecidos 
seria o $ponto flutuante$ que ainda hoje é o mais usado e que dependendo da implementação
é bastante confiável, com um baixo erro nos resíduos das operação. A IEEE padronizou
a ponto flutuante para diminuir os erros de operação, quando vários programas 
diferente são usados para resolver um problema em comum. Mas esta técnica não
é a única, existe no mundo dos dispositivos embarcados a conhecida técnica do 
ponto fixo, que neste trabalho será explorada e desenvolvida. O ponto fixo 
não o mais recomendado para operações que requerem uma alta precisão. Mais 
pode ser usado em operações que o erro residual na multiplicação não seja
algo que atrapalhe aplicação ou equipamento de funcionarem coerentemente.

