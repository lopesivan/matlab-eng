% Projeto de Digital
% Felipe Bandeira da Silva

\documentclass[a4paper, 11pt]{article}
%\documentclass[paper=a4, fontsize=11pt]{article}
%\usepackage[a4paper,pdftex]{geometry}										% A4paper margins
%\setlength{\oddsidemargin}{5mm}												% Remove 'twosided' indentation
%\setlength{\evensidemargin}{5mm}


\usepackage[brazil]{babel}
\usepackage[utf8]{inputenc}
\usepackage{listings}
\usepackage{color}
\usepackage{amsthm}
\usepackage{graphicx}

\title{Projeto de Eletrônica Digital\\Computador 4 bits}


%%%%%%%%%%%%%%%%%%%%%%%%%%%%%%%%%%%%%%%%%%%%%%%%%%%%%%%%%%%%%%%%%%%%%%%%%%%%%%%%
% MAIN
%%%%%%%%%%%%%%%%%%%%%%%%%%%%%%%%%%%%%%%%%%%%%%%%%%%%%%%%%%%%%%%%%%%%%%%%%%%%%%%%

\begin{document}

\maketitle

%\include{pagina_inicial}
%\newpage
%\section{Introdução}

Durante o desenvolvimento dos computadores digitais o maior problema encontrado
é o tempo de processamento. Com a criação de novas necessidades para a raça humanas
sejam elas essenciais ou apenas para o lazer, os computadores se tornaram 
itens obrigatório, e quanto mais rápido ele são, mais tarefas podem desempenhar.
Um operação que requer um quantidade de processamento considerável é a multiplicação,
uma operação que surge naturalmente em qualquer modelo matemático. Diversos
métodos digitais foram desenvolvidos para está operação. Um dos mais conhecidos 
seria o $ponto flutuante$ que ainda hoje é o mais usado e que dependendo da implementação
é bastante confiável, com um baixo erro nos resíduos das operação. A IEEE padronizou
a ponto flutuante para diminuir os erros de operação, quando vários programas 
diferente são usados para resolver um problema em comum. Mas esta técnica não
é a única, existe no mundo dos dispositivos embarcados a conhecida técnica do 
ponto fixo, que neste trabalho será explorada e desenvolvida. O ponto fixo 
não o mais recomendado para operações que requerem uma alta precisão. Mais 
pode ser usado em operações que o erro residual na multiplicação não seja
algo que atrapalhe aplicação ou equipamento de funcionarem coerentemente.




%%%%%%%%%%%%%%%%%%%%%%%%%%%%%%%%%%%%%%%%%%%%%%%%%%%%%%%%%%%%%%%%%%%%%%%%%%%%%%%%
% INTRODUÇÃO
%%%%%%%%%%%%%%%%%%%%%%%%%%%%%%%%%%%%%%%%%%%%%%%%%%%%%%%%%%%%%%%%%%%%%%%%%%%%%%%%

\newpage

\section{Introdução}

Os computadores surgiram muito antes que o ser humano pensassem neles 
de forma que são hoje. O inicio foi um tanto dramático, no século IV 
antes de cristo, os Gregos iniciavam uma nova matemática. Uma matemática
menos aplicada aos problemas da vida real, uma matemática abstrata que
tratava de problemas que fugiam das explicações obvias, com um conteúdo
de imaginação e surrealismo. Os gregos iniciaram o desenvolvimento da
lógica do certo ou errado. Com o passar dos tempos um matemático
Boolen fomentou o logica binária que mais tarde seria a base dos computadores
atuais. 

Esse trabalho tem como principal objetivo mostra o processo de desenvolvimento
e montagem de uma unidade logica aritmética(ULA) de 4 bits. 

%%%%%%%%%%%%%%%%%%%%%%%%%%%%%%%%%%%%%%%%%%%%%%%%%%%%%%%%%%%%%%%%%%%%%%%%%%%%%%%%
% 
%%%%%%%%%%%%%%%%%%%%%%%%%%%%%%%%%%%%%%%%%%%%%%%%%%%%%%%%%%%%%%%%%%%%%%%%%%%%%%%%
\newpage

\section{ULA}

A base para o projeto é a ULA construída pelo SN54/74LS181



\end{document}
