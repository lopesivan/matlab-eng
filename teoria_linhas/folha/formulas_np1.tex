\documentclass[a4paper, 10pt, twocolumn]{article}
%\documentclass[a4paper, 10pt]{article}
\usepackage[brazil]{babel}
\usepackage[utf8]{inputenc}
\usepackage{amsthm}
\usepackage{graphicx}

\newcommand{\unit}[1]{\ensuremath{\, \mathrm{#1}}}

\title{Formulário\\Linhas e Transmissão}
\author{Felipe Bandeira da Silva}

\begin{document}
\maketitle
\section{Modelagem de Carga}

A modelagem de uma carga elétrica pode ser feita para:

\begin{itemize}
\item Potência ativa constante
\item Impedância constante
\item Carga mista
\end{itemize}

\subsection{Potência ativa constante(PCTE)}
    
\begin{equation}
P = VA \cdot fp
\end{equation}

\begin{equation}   
Q_{modelada} = P \cdot tan(cos^{-1}fp)
\end{equation}

\begin{equation}
S = P+jQ_{modelada}
\end{equation}

\subsection{Impedância Constante (ZCTE)}

\begin{equation}
P_{modelada} = P \cdot \frac{V_{barramento}}{V_{carga}}
\end{equation}

\begin{equation}   
S = P_{modelada} + jQ_{modelada}  
\end{equation}

\subsection{Carga Mista}

\begin{equation}
P_{modelada} =PCTE \cdot P +  ZCTE \cdot  P \cdot \frac{V_{barramento}}{V_{carga}}
\end{equation}

\begin{equation}
S = P_{modelada} + jQ_{modelada}  
\end{equation}

\section{Indutância em LT's}

A indutância de uma linha de transmissão é relacionada pelo
comprimento da linha, área efetiva do condução e o efeito da 
indutância mútua entre as linhas. Os cálculos apresentado nas
próximas secções são válidos para linhas monofásicas e trifásicas. 

\subsection{Linha monofásica}

\begin{equation}
L = 2 \cdot 10^{-4} \ln\frac{D}{R'} \unit{H/km}
\end{equation}
    
\begin{equation}
L = 2 \cdot 10^{-7} \ln\frac{D}{R'} \unit{H/m}
\end{equation}

\begin{equation}
R' = e^{-\frac{1}{4}}R = 0.7788R
\end{equation}

\subsection{Linha trifásica}
Indutância final de uma linha trifásica com $n$ condutores por fase é data por:

\begin{equation}
L = 2 \cdot 10^{-4} \ln\frac{D_m}{D_s}
\end{equation}

Onde $D_m$ é a distância média geométrica entre as fases,

\begin{equation}
D_m = (D_{12} D_{21} D_{31})^{\frac{1}{3}}
\end{equation}

$D_s$ é definida para os $n$ condutores por fase,
\begin{center}
\begin{tabular}{|c||c|}
    \hline
    Número de condutores & $D_s$ \\
    \hline
    1 & $R'$ \\
    2 & $(R' d)^{\frac{1}{2}}$ \\
    3 & $(R' d^2)^{\frac{1}{3}}$ \\
    4 & $(2^{\frac{1}{2}} R' d^2)^{\frac{1}{4}}$ \\ 
    \hline
\end{tabular}
\end{center}

Considerando o espaçamento $d$ igual entre os condutores que estão na mesma coluna ou linha.

\end{document}

