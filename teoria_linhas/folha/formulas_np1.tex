\documentclass[a4paper, 10pt, twocolumn]{article}
\usepackage[brazil]{babel}
\usepackage[utf8]{inputenc}
\usepackage{amsthm}


\newcommand{\unit}[1]{\ensuremath{\, \mathrm{#1}}}

\title{Linhas e Transmissão}
\author{Felipe Bandeira da Silva\\Fortaleza-CE}

\begin{document}
\maketitle
\section{Modelagem de Carga}

A modelagem de uma carga elétrica pode ser:

\begin{itemize}
\item Potência ativa constante
\item Impedância constante
\item Carga mista
\end{itemize}

\subsection{Potência ativa constante(PCTE)}
    
    $$
        P = VA \cdot fp
    $$
    
    $$
        Q_{modelada} = P \cdot tan(cos^{-1}fp)
    $$
    
    $$
        S = P+jQ_{modelada}
    $$

\subsection{Impedância Constante (ZCTE)}
    $$
        P_{modelada} = P \cdot \frac{V_{barramento}}{V_{carga}}
    $$
    
    $$
        S = P_{modelada} + jQ_{modelada}  
    $$

\subsection{Carga Mista}
    $$
        P_{modelada} =PCTE \cdot P +  ZCTE \cdot  P \cdot \frac{V_{barramento}}{V_{carga}}
    $$

    $$
       S = P_{modelada} + jQ_{modelada}  
    $$

\section{Indutância em LT's}

Indutância L de uma linha de transmissão, desprezando a altura
da linha e não linearidade do meio.

\subsection{Linha monofásica}

    $$
        L = 2 \cdot 10^{-4} ln\frac{D}{R'} \unit{H/km}
    $$
    
    $$
        L = 2 \cdot 10^{-7} ln\frac{D}{R'} \unit{H/m}
    $$

    $$
        R' = e^{-\frac{1}{4}}R = 0.7788R
    $$

\subsection{Linha trifásica}
Indutância final de uma linha trifásica com $n$ condutores por fase é data por:

$$
    L = 2 \cdot 10^{-4} ln\frac{D_m}{D_s}
$$

Onde $Dm$ é a distância média geométrica entre as fases,

$$
D_m = (D_{12} D_{21} D_{31})^{\frac{1}{3}}
$$

E $D_s$ é definida para os $n$ condutores por fase,



\end{document}

