% Modelo de linha longa, parâmetros distribuidos
% Felipe Bandeira da Silva
% 27/09/2013

\documentclass[a4paper, 10pt]{article}
%\documentclass[paper=a4, fontsize=11pt]{article}

\usepackage[brazil]{babel}
\usepackage[utf8]{inputenc}
\usepackage{listings}
\usepackage{color}
\usepackage{amsthm}
\usepackage{graphicx}

\usepackage{schemabloc}
\usetikzlibrary{circuits}

\usepackage{circuitikz}
\usepackage{tikz}

\setlength{\parindent}{0pt}
\setlength{\parskip}{18pt}

\title{Linha de Transmissão Longa}
\author{Felipe Bandeira da Silva}

%\date{}

\begin{document}

\maketitle

%\newpage

Será utilizada o modelo de parâmetro distribuído, onde:\\
R - Resistência da LT\\
L - Indutância da LT\\
C - Capacitância da LT\\
G - Condutância do AR

\begin{center}
    \begin{circuitikz}[scale=.8] \draw
    	(0, 0) to[R=$R\Delta_x$] (2, 0)
        (2, 0) to[L=$$, i=$I$] (4, 0)
        (4, 0) to[R=$G \Delta_x$] (4, -2)
        (6, 0) to[C=$C \Delta_x$] (6, -2)
        (4, 0) to[short] (6, 0)
        (6, 0) to[short, i=$I+\Delta I$] (8, 0)
        (0, -2) to[short] (8, -2)
        node[ocirc] (A) at (0,0) {}
        node[ocirc] (B) at (0,-2) {}
        (A) to [open, v=$V$] (B)
        node[ocirc] (C) at (8,0) {}
        node[ocirc] (D) at (8,-2) {}
        (C) to [open, v^=$V+\Delta_V$] (D)       
        ;
    \end{circuitikz}
\end{center}

Pela lei das malhas,

\begin{equation}
V = (R \Delta_x) I + (L \Delta_x) \frac{\partial I}{\partial t} + V + \Delta V
\end{equation}

Desenvolvendo 1,

\begin{equation}
\Delta V = -(R \Delta_x) I - (L \Delta_x) \frac{\partial I}{\partial t}
\end{equation}

Pela lei dos nós,

\begin{equation}
I = (G \Delta_x)(V + \Delta V) + (C \Delta_x) \frac{\partial ( v +  \Delta V)}{\partial t} + I + \Delta I
\end{equation}

\begin{equation}
\Delta I = (G \Delta_x)(V + \Delta V) - (C \Delta_x) \frac{\partial (v + \Delta V)}{\partial t}
\end{equation}

\begin{equation}
\Delta I = - (G \Delta_x V) - (G \Delta_x \Delta V) - (C \Delta_x) \frac{\partial V}{\partial t} - (C \Delta V) \frac{\partial \Delta V}{\partial t}
\end{equation}

%\newpage 

Substituindo o valor de $\Delta V$ dado pela equação (2) na equação (5) temos,

\begin{eqnarray}
\lefteqn{\Delta I=}\nonumber\\ 
& &-(G \Delta_x V)-(G \Delta_x)(-(R \Delta_x)I - (L \Delta_x) \frac{\partial I}{\partial t})\nonumber\\
& &-(C \Delta_x)\frac{\partial V}{\partial I} - (C \Delta_x) \frac{\partial}{\partial t}\left( -(R \Delta_x)I - (L \Delta_x) \frac{\partial I}{\partial t}\right)
\end{eqnarray}

Simplificando,

\begin{equation}
\Delta I = -(G \Delta_x V) + G R \Delta_x^2 I + G L \Delta_x^2 \frac{\partial I}{\partial t} - C \Delta_x \frac{\partial I}{\partial t} + C \Delta_x^2 R \frac{\partial I}{\partial t} + C L \Delta_x^2 \frac{\partial^2 I}{\partial t^2}
\end{equation}

Dividindo a equação 2 por $\Delta_x$, temos,

\begin{equation}
\frac{\Delta V}{\Delta x} = - R I - L \frac{\partial I}{\partial t}
\end{equation}

Aplicando o limite na equação 8, e tendendo $\Delta_x \to 0$ temos,

\begin{equation}
\lim_{\Delta \to 0}\frac{\Delta V}{\Delta x} = - \lim_{\Delta \to 0}{R I} - \lim_{\Delta \to 0}{L \frac{ \partial I}{\partial t}}
\end{equation}

\begin{equation}
\frac{\partial V}{\partial x} = - R I - L \frac{\partial I}{\partial t}
\end{equation}

Dividindo a equação 7 por $\Delta_x$ temos,

\begin{equation}
\frac{\Delta I}{\Delta x} = -G V + G R \Delta_x I + G L \Delta_x \frac{\partial I}{\partial t} - C \frac{\partial V}{\partial t} + C L \Delta_x \frac{\partial^2 I}{\partial t^2}
\end{equation}

%\newpage

Aplicando o limite na equação 11 e tendendo $\Delta_x \to 0$ temos,

\begin{eqnarray}
\lefteqn{\lim_{\Delta \to 0}{\frac{\Delta I}{\Delta x}} =}\nonumber\\
& &- \lim_{\Delta \to 0}{(GV)} + \lim_{\Delta \to 0}{(G R \Delta_x I)} + \lim_{\Delta \to 0}{\left(G L \Delta_x \frac{\partial I}{\partial t}\right)}\nonumber\\ 
& &- \lim_{\Delta \to 0}{\left( C \frac{\partial C}{\partial t}\right)} + \lim_{\Delta \to 0}{\left(C L \Delta_x \frac{\partial^2 I}{\partial t^2}\right)}
\end{eqnarray}

Simplificando,

\begin{equation}
\frac{\partial I}{\partial x} = - G V - C \frac{\partial V}{\partial t}
\end{equation}

As equações 10 e 13 são chamadas de equações fundamentais de propagação.
Derivando a equação 10 em relação a $x$, temos,

\begin{equation}
\frac{\partial^2 V}{\partial x^2} = -R \frac{\partial I}{\partial x} - L \frac{\partial \frac{\partial I}{\partial t}}{\partial t}
\end{equation}

Substituindo o valor de $\frac{\partial I}{\partial x}$ dado pela equação
13 na equação 14, temos,

\begin{eqnarray}
\frac{\partial^2 V}{\partial x^2} &=& - R \left( - G V - C \frac{\partial V}{\partial t}\right) - L \frac{\partial -G V - L \frac{\partial V}{\partial t}}{\partial t} \\
&=&R G V + R C \frac{\partial V}{\partial t} + L G \frac{\partial V}{\partial t} + L C \frac{\partial^2 V}{\partial t^2} \\
&=&R G V + (R C + L G)\frac{\partial V}{\partial t} + L C \frac{\partial^2}{\partial t^2}
\end{eqnarray}

Derivando a equação 13 em relação a ''x'', temos,

\begin{equation}
\frac{\partial^2 I}{\partial x^2} =  - G \frac{\partial V}{\partial t} - C \frac{\partial }{\partial t}\frac{\partial V}{\partial t}
\end{equation}

Substituindo o valor de $\frac{\partial V}{\partial x}$ dada pela equação
10 na equação 18, temos,

\begin{eqnarray}
\frac{\partial^2 I}{\partial x^2} &=& -G \left( -R I -L \frac{\partial I}{\partial t}\right) - L \frac{\partial}{\partial t}\left(-R I - L \frac{\partial I}{\partial t}\right) \\
&=& R G I + L G \frac{\partial I}{\partial t} + R C \frac{\partial I}{\partial t} + L C \frac{\partial^2 I}{\partial t^2}\\
&=& R G I + (R C + L G)\frac{\partial I}{\partial t} + L C \frac{\partial^2 I}{\partial t^2}
\end{eqnarray}

Antes de resolver as equações 17 e 21, se faz necessário,

\begin{equation}
V = \sqrt{2} V_{ef} e^{j \omega t}
\end{equation}

\begin{equation}
I = \sqrt{2} I_{ef} e^{j \omega t}
\end{equation}

Derivando,

\begin{equation}
\frac{\partial V}{\partial t} = j \omega \sqrt{2} V_{ef} e^{j \omega t}
\end{equation}

Logo,

$$
\frac{\partial V}{\partial t} = j \omega V
$$

Derivando novamente,

\begin{equation}
\frac{\partial^2 V}{\partial t^2} = j \omega j \omega \sqrt{2} V_{ef} e^{j \omega t}
\end{equation}

Logo,

$$
\frac{\partial^2 V}{\partial t^2} = j \omega j \omega V
$$

Aplicando para a corrente,

\begin{equation}
\frac{\partial I}{\partial t} = j \omega \sqrt{2} I_{ef} e^{j \omega t}
\end{equation}

Logo,

$$
\frac{\partial I}{\partial t} = j \omega I
$$

Desenvolvendo a equação 17, temos,

\begin{eqnarray}
\frac{\partial^2 V}{\partial x^2} &=& R G V + (R C + L G)j \omega V + L C j \omega j \omega V \nonumber\\
&=&R G V + (R j \omega C + G j \omega L) v + j \omega L j \omega C V \nonumber\\
&=&R G V + V (R(j \omega C) +  G(j \omega L)) + (j \omega L)(j \omega C) V\nonumber\\
&=&V \left( (R + j \omega L) (G + j \omega C))\right) 
\end{eqnarray}

Desprezando o valor da susceptância, $G = 0$ e lembrando $R+j \omega L = Z$

\begin{equation}
\frac{\partial^2 V}{\partial x^2} - V Z Y = 0
\end{equation}

Seja $\frac{d}{d x} = D$ logo $\frac{d^2}{d x^2} = D^2$ é possivel desenvolver a equação 28 até,

\begin{equation}
D^2 V - Z V Y = 0
\end{equation}

Temos,

\begin{equation}
(D^2 - Z Y)V = 0
\end{equation}

Com $V \neq 0$,

\begin{equation}
D^2 - Z Y = 0
\end{equation}

Sendo soluções de 31,

\begin{equation}
D_1 = \sqrt{Z Y}
\end{equation}

\begin{equation}
D_2 = -\sqrt{Z Y}
\end{equation}

A solução da equação diferencial é,

\begin{equation}
V = A_1 e^{\sqrt{Z Y} x} + A_2 e^{-\sqrt{Z Y} x}
\end{equation}

Comprovando que a equação 34 é solução para a equação 17,

\begin{eqnarray*}
\frac{\partial V}{\partial x} = \sqrt{Z Y} A_1 e^{\sqrt{Z Y}x} - \sqrt{Z Y} A_2 e^{-\sqrt{Z Y}x}
\end{eqnarray*}

\begin{eqnarray}
\frac{\partial^2 V}{\partial x^2} &=& \sqrt{Z Y} \sqrt{Z Y} A_1 e^{\sqrt{Z Y}x} + \sqrt{Z Y} \sqrt{Z Y} A_2 e^{-\sqrt{Z Y}x} \nonumber\\
&=& Z Y A_1 e^{\sqrt{Z Y}x} + Z Y A_2 e^{-\sqrt{Z Y}x}\nonumber\\
&=& Z Y \left( A_1 e^{\sqrt{Z Y}x} + Z Y A_2 e^{\sqrt{Z Y}x} \right)\nonumber\\
&=& Z Y V
\end{eqnarray}

Com a equação 35 se prova a solução da equação 34.

Sabendo que,

\begin{equation}
\frac{\partial V}{\partial x} = - R I - L j \omega I = -I (R + j \omega L)
\end{equation}

Logo,

$$
\frac{\partial V}{\partial x} = - I Z
$$

Para tanto,

\begin{equation}
\sqrt{Z Y} A_1 e^{\sqrt{Z Y} x} - \sqrt{Z Y} A_2 e^{-\sqrt{Z Y} x} = - I Z
\end{equation}

\begin{equation}
-I = \frac{\sqrt{Z Y}}{Z} A_1 e^{\sqrt{Z Y} x} - \frac{\sqrt{Z Y}}{Z} A_2 e^{-\sqrt{Z Y} x} 
\end{equation}

Sabendo que,

\begin{equation}
\frac{\sqrt{Z Y}}{Z} = \frac{1}{\sqrt{\frac{Z}{Y}}}
\end{equation}

Rearranjando a equação 38,

\begin{equation}
-I = \frac{1}{\sqrt{\frac{Z}{Y}}} A_1 e^{\sqrt{Z Y} x} - \frac{1}{\sqrt{\frac{Z}{Y}}} A_2 e^{-\sqrt{Z Y} x} 
\end{equation}

Finalmente,

\begin{equation}
V = A_1 e^{\gamma x} +   A_2 e^{- \gamma x}
\end{equation}

\begin{equation}
I = \frac{1}{Z_C} A_1 e^{\gamma x} +  \frac{1}{Z_C} A_2 e^{- \gamma x}
\end{equation}

Onde, $\gamma$ é conhecido como constante de propagação sendo um número
adimensional, $Z_C$ é conhecida como impedância caractéristica e é 
independente do comprimento da linha.

Para, $x=0$ faz $V=V_R$

\begin{equation}
V_T^{FN} = A_1 + A_2
\end{equation}

\begin{equation}
I_T = \frac{1}{Z_C} A_1 - \frac{1}{Z_C} A_2
\end{equation}

Os valores de $A_1$ e $A_2$ são,

\begin{equation}
A_1 = \frac{V_R^{FN} +  Z_C I_R}{2}
\end{equation}

\begin{equation}
A_2 = \frac{V_R^{FN} -  Z_C I_R}{2}
\end{equation}

Substituindo 45 e 46 em 41 e 42,

\begin{equation}
V_T^{FN} = \frac{V_R^{FN} +  Z_C I_R}{2} e^{\gamma x} +   \frac{V_R^{FN} -  Z_C I_R}{2} e^{- \gamma x}
\end{equation}

\begin{equation}
I_T = \frac{1}{Z_C} \frac{V_R^{FN} +  Z_C I_R}{2} e^{\gamma x} +  \frac{1}{Z_C} \frac{V_R^{FN} -  Z_C I_R}{2} e^{- \gamma x}
\end{equation}

Pela teoria dos quadripolos,

\begin{equation}
V_T^{FN} = A V_R^{FN} + B I_R
\end{equation}

\begin{equation}
I_T = C V_R^{FN} + D I_R
\end{equation}

Identificando os parametros dos quadripolos,

\begin{equation}
V_T^{FN} = \frac{V_R^{FN}}{2} e^{\gamma x} + \frac{Z_C I_R}{2} e^{\gamma x} +  \frac{V_R^{FN}}{2} e^{-\gamma x} - \frac{Z_C I_R}{2} e^{-\gamma x}
\end{equation}

Rearranjando os termos,

\begin{equation}
V_T^{FN} = \left( \frac{e^{\gamma x} + e^{-\gamma x}}{2}\right) V_R^{FN} + Z_C \left( \frac{e^{\gamma x} - e^{-\gamma x}}{2}\right) I_R
\end{equation}

Sabe-se que,

\begin{equation}
\cosh{\gamma x} = \left( \frac{e^{\gamma x} + e^{-\gamma x}}{2}\right)
\end{equation}

\begin{equation}
\sinh{\gamma x} = \left( \frac{e^{\gamma x} - e^{-\gamma x}}{2}\right)
\end{equation}

Então a equação 52 fica,

\begin{equation}
V_T^{FN} = \cosh{(\gamma x)} V_R^{FN} + \sinh{(\gamma x)}  Z_C I_R 
\end{equation}

Para a corrente,

\begin{equation}
I_T = \left( \frac{\frac{V_R^{FN}}{Z_C} + I_R}{2}\right) e^{\gamma x} - \left( \frac{\frac{V_R^{FN}}{Z_C} - I_R}{2}\right) e^{-\gamma x} 
\end{equation}

Rearranjando os termos,

\begin{equation}
I_T = \frac{V_R^{FN}}{2 Z_C} e^{\gamma x} + \frac{I_R}{2} e^{\gamma x} - \frac{V_R^{FN}}{2 Z_C} e^{-\gamma x} + \frac{I_R}{2} e^{-\gamma x}
\end{equation}

Para a identificação,

\begin{equation}
I_T = \frac{1}{Z_C} \left( \frac{e^{\gamma x} - e^{-\gamma x}}{2}\right) V_R^{FN} + \left( \frac{e^{\gamma x} + e^{-\gamma x}}{2}\right) I_R
\end{equation}

Portanto,

\begin{equation}
I_T = \frac{1}{Z_C} \sinh{(\gamma x)}  V_R^{FN} + \cosh{(\gamma x)} I_R
\end{equation}

Em resumo os parâmetros do quadripolos para uma linha longa são,

\renewcommand{\arraystretch}{1.5}
\begin{center}
\begin{tabular}{c||c}
	A & $\cosh{\gamma x}$ \\
    B & $Z_C \sinh{\gamma x}$ \\
    C & $\frac{\sinh{\gamma x}}{Z_C}$ \\
    D & $\cosh{\gamma x}$ \\
\end{tabular}
\end{center}

Antes de finalizar, se faz necessário provar que $A D - B C = 1$

\begin{equation}
 \cosh{\gamma x} \cosh{\gamma x} - Z_C \sinh{\gamma x} \frac{\sinh{\gamma x}}{Z_C} = 1
\end{equation}

Portanto,

\begin{equation}
 \cosh^2{(\gamma x)} - \sinh^2{(\gamma x)} = 1
\end{equation}

\end{document}
