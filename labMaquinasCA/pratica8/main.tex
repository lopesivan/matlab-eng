%\documentclass[a4paper, 10pt]{article}
\documentclass[paper=a4, fontsize=11pt]{article}


\usepackage[brazil]{babel}
\usepackage[utf8]{inputenc}
\usepackage{listings}
\usepackage{color}
\usepackage{amsthm}
\usepackage{graphicx}
\usepackage{cite}

\usepackage{schemabloc}
\usetikzlibrary{circuits}

\usepackage{tabularx,ragged2e,booktabs,caption}

\usepackage{hyperref}

\setlength{\parindent}{0pt}
\setlength{\parskip}{18pt}

\title{Laboratório de Máquinas de Corrente Alternada\\Motor Monofásico com Capacitor Permanente.}
\author{Felipe Bandeira da Silva\\1020942-X}
%\date{}

\begin{document}


\maketitle

%\newpage

%\begin{abstract}
\textit{Este laboratório tem como objetivo: Analisar a estrutura do motor de operação continua por capacitor. Determinar suas características de partida e operação contínua. Comparar estas características com as dos motores monofásicos com partida por capacitor e de fase dividida.}
%\end{abstract}

\newpage

\tableofcontents

\newpage

%\listoffigures


%%%%%%%%%%%%%%%%%%%%%%%%%%%%%%%%%%%%%%%%%%%%%%%%%%%%%%%%%%%%%%%%%%%%%%%%%%%%%%%%
% fundamentação teórica
\newpage
\section{Introdução}

O motor aqui estudado, como todos os outros, é empregado em ventiladores, 
compressores, bombas, talhas, guinchos, transportadoras, alimentadoras de uso
rural, trituradores, bombas para adubação, descarregamentos de silos e 
outras de uso geral. A WEG disponibiliza estes motores com potência 
que vão de 1/12 a 3/4 de CV, tensão de alimentação 127 ou 220 Volts.

Em comparação com os motores trifásicos os motores monofásicos 
apresentam muitas desvantagens:

\begin{itemize}
    \item apresentam maiores volume e peso para potências e velocidades 
            iguais.
    \item necessitam de manutêncão mais apurada devido ao circuito de 
            partida e seus acessórios.
    \item apresentam rendimento e fator de potência menores para a mesma potência.
    \item possuem menor conjugado de partida
    \item são difíceis de encontrar no comércio para potências mais elevadas.
\end{itemize}

O modo de partida do motor monofásico de capacitor permanente funciona da 
seguinte forma. O enrolamento auxiliar e seu capacitor em série ficam permanentemente
conectados, não sendo necessária a chave centrífuga. Isto é bom porquê a 
ausência de partes móveis facilita a manutenção.
O conjugado máximo, o rendimento e o fator de potência desses motores são 
melhores que os de outros tipos, aproximado-se aos valores obtidos em 
motores trifásicos. Em contrapartida, seu conjugado de partida  é menor
que o dos motores de fase divida, limitando sua utilização a equipamentos 
com pequenas serras, furadeiras e etc.

\section{Parte Pratica}

\textit{A seguir perguntas referentes ao estato físico do motor}.
\newline
Observando o motor a partir da parte frontal do módulo, 

\begin{itemize}
        \item Os dois enrolamentos do estator se compõem de numerosas espiras. Identifique-os.
        \item Os enrolamentos do estator parecem idênticos? \textbf{SIM}.
        \item Os dois enrolamentos do estator estão montados exatamente um sobre
                o outro? \textbf{SIM}.
        \item Quantos pólos existem? \textbf{4}.
\end{itemize}

\subsection{Ligações básicas e testes iniciais}

Usando o módulo EMS e fazendo as ligações mostradas no manual 
do laboratório. A tabela 9.1 foi construída e é mostrada abaixo.

\begin{center}
    \begin{tabular}{c|c|c|c|c}
            Conjungado (lbf.in) & I (Amperes) & VA & P (Watts) & Velocidade (rpm) \\
            \hline
            0 & 1.05 & 127 & 123 & 1787 \\
            3 & 1.40 & 167 & 167 & 1772 \\
            6 & 2.01 & 238 & 234 & 1742 \\
            9 & 2.50 & 295 & 292 & 1712 \\
    \end{tabular}
    \captionof{table}{}
\end{center}

Determine o conjugado máximo de partida que o motor desenvolve em 
operação contínua por capacitor. 

O conjugado encontrado foi de $4.5$ lbf.in. Com muita relutância na
partida.

A corrente de partida encontrada para esta situação foi de $6.9$ Amperes.

\section{Conclusões}

O motor de capacitor de partida permanente é outro motor monofásico
com a vantagem de não mais precisar de um chave centrifuga. Tem o 
mesmo emprego de um motor monofásico comum, não sendo diferente entre
a sua categoria. Entretanto como qualquer outro motor monofásico 
quando comparado a um motor trifásico é mais desvantajoso.


\end{document}
