%\documentclass[a4paper, 10pt]{article}
\documentclass[paper=a4, fontsize=11pt]{article}


\usepackage[brazil]{babel}
\usepackage[utf8]{inputenc}
\usepackage{listings}
\usepackage{color}
\usepackage{amsthm}
\usepackage{graphicx}
\usepackage{cite}

\usepackage{schemabloc}
\usetikzlibrary{circuits}

\usepackage{tabularx,ragged2e,booktabs,caption}

\usepackage{hyperref}

\setlength{\parindent}{0pt}
\setlength{\parskip}{18pt}

\title{Laboratório de Máquinas de Corrente Alternada\\Motor Universal.}
\author{Felipe Bandeira da Silva\\1020942-X}
%\date{}

\begin{document}


\maketitle

%\newpage

%\begin{abstract}
\textit{Este laboratório tem como objetivo: Analisar a estrutura do motor universal. Determinar as características em vazio e a plena carga quando funciona com corrente alternada. Determinar suas características em vazio e a plena carga quando funciona com corrente continua.}
%\end{abstract}

\newpage

\tableofcontents

\newpage

%\listoffigures


%%%%%%%%%%%%%%%%%%%%%%%%%%%%%%%%%%%%%%%%%%%%%%%%%%%%%%%%%%%%%%%%%%%%%%%%%%%%%%%%
% fundamentação teórica
\newpage
\section{Introdução}

O motor universal é um motor que trabalha com tensão alternada ou 
continua. O dispositivo utilizado para a comutação entre AC e CC é 
realizado por um comutador em série com a bobina no estator.
Os motores universais tem um alto torque inicial, apresentam 
uma alta velocidade, são pequenos e normalmente usados em 
equipamento pequenos, comumente em eletrodomésticos. Entretanto é um 
motor que produz muito barulho devido ao comutador universal.

Quando funcionando com corrente alternada a rotação do motor é 
diretamente dependente da frequência da fonte de alimentação. A
eficiência do motor quando comparado com motores de indução tipo 
gaiola de esquilo é da ordem de 30 $\%$ menor. Para motores de 
potência superiores a eficiência chega a ser 70 a 75 $\%$ menor.

Motores universais tem velocidade de 4000 a 16000 rpm podendo 
chegar a 20 mil rpm. Caso a velocidade supere as velocidade 
estabelecida no projeto o motor pode ser danificado.

O motor universal é fundamentalmente um motor CC projetado especialmente
para funcionar com CA e com CC. Um motor série normal CC funciona 
muito deficientemente em CA. Devido sobretudo a duas razões,

\begin{itemize}
        \item A alta reatância dos enrolamentos de armadura e campo
                limita a corrente CA a um valor muito menor que a corrente
                contínua, para a mesma tensão de linha.
        \item Quando se uso ferro sólido para o motor ou carcaça do estator, 
                o fluxo de CA produzirá grandes correntes parasitas e, 
                portanto aquecerá.
\end{itemize}

\section{Prática}

A parte prática consiste no inicio, é: Examine a estrutura do módulo
do motor universal EMS 8254, dando especial atenção ao motor, às escovas
e aos terminais de conexão. 

Observando o motor pela parte posterior:

\begin{itemize}
    \item O estator nos pinos, 1 e 2
    \item Enrolamento principal, 3 e 4
    \item Enrolamento de compensação, 5 e 6
\end{itemize}

A tensão aumenta e logo diminuiu quando as escovas se aproximam 
da outra posição extrema.

\subsection{Alimentação CA}

Ligando o enrolamento da armadura e compensação em série com
a saída de 0 a 120 VCA da fonte de alimentação. 

A corrente de linha foi de 1.52 Amperes.

\subsection{Dinamômetro, com CA}

Usando o dinamômetro acoplado com o motor universal, foi construído 
a seguinte tabela. Não sendo pego o valor de 9 lbf.in já que o 
motor não suporta tal carga. E alimentando o motor com corrente 
alternada.

\begin{center}
    \begin{tabular}{c|c|c|c|c}
            Conjungado (lbf.in) & I (Amperes) & VA & P (Watts) & Velocidade (rpm) \\
            \hline
            0 & 1.22 & 146.40 & 148.00 & 3740 \\
            3 & 1.66 & 199.20 & 195.00 & 2640 \\
            6 & 2.28 & 273.60 & 259.00 & 1865 \\
    \end{tabular}
    \captionof{table}{}
\end{center}

\subsection{Dinamômetro, com CC}

Usando o dinamômetro acoplado com o motor universal, foi construído 
a seguinte tabela. Não sendo pego o valor de 9 lbf.in já que o 
motor não suporta tal carga. E alimentando o motor com corrente 
continua.

\begin{center}
    \begin{tabular}{c|c|c|c|c}
            Conjungado (lbf.in) & I (Amperes) & VA & P (Watts) & Velocidade (rpm) \\
            \hline
            0 & 1.27 & - & 152.40 & 4066 \\
            3 & 1.73 & - & 207.60 & 2676 \\
            6 & 2.15 & - & 258.00 & 2050 \\
    \end{tabular}
    \captionof{table}{}
\end{center}


\section{Conclusão}

O motor universal é uma maquina incrível, é capaz de funcionar com 
corrente alternada como com corrente continua. Não perdendo as suas 
principais características. Que são alta velocidade de rotação, 
muito barulho devido ao comutador de CA e CC, alto torque inicial.
Durante toda a prático o motor utilizado apresentou um barulho 
bem tipico e bem alto, chegando a incomodar quando se coloca mais
carga. 

\end{document}
