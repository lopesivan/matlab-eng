%\documentclass[a4paper, 10pt]{article}
\documentclass[paper=a4, fontsize=11pt]{article}


\usepackage[brazil]{babel}
\usepackage[utf8]{inputenc}
\usepackage{listings}
\usepackage{color}
\usepackage{amsthm}
\usepackage{graphicx}
\usepackage{cite}

\usepackage{schemabloc}
\usetikzlibrary{circuits}

\usepackage{tabularx,ragged2e,booktabs,caption}

\usepackage{hyperref}

\setlength{\parindent}{0pt}
\setlength{\parskip}{18pt}

\title{Laboratório de Máquinas de Corrente Alternada\\Motor Monofásico de Indução Partida a Capacitor.}
\author{Felipe Bandeira da Silva\\1020942-X}
%\date{}

\begin{document}


\maketitle

%\newpage

%\begin{abstract}
\textit{Este laboratório tem como objetivo: Medir as características de partida
e funcionamento do motor com partida por capacitor. Comparar seu funcionamento
durante a patida e operação contínua com o do motor monofásico de fase dividida.}
%\end{abstract}

\newpage

\tableofcontents

\newpage

%\listoffigures


%%%%%%%%%%%%%%%%%%%%%%%%%%%%%%%%%%%%%%%%%%%%%%%%%%%%%%%%%%%%%%%%%%%%%%%%%%%%%%%%
% fundamentação teórica
\newpage
\section{Introdução}

Quando se interrompe uma das fases de um motor trifásico enquanto ele
será girando, este continua a funcionar como motor monofásico suportando
a carga, ainda que a menor velocidade. Entretanto, se o motor estiver
parado, não poderia arrancar porque faltaria o campo magnético rotativo
e não teria partida automática. 

Para que o motor possa funcionar é 
necessário fazer o arranque por algum meio exterior, e então, a reação
do rotor, provoca o nascimento de um campo magnético rotativo.
Assim como o motor trifásico não arranca, sozinho quando lhe falta
uma fase, um motor monofásico para ser posto em marcha precisa também 
utilizar-se de um dispositivo auxiliar.

O sistema de partida por capacitor é uma modificação do método de 
fase divida, e utiliza um capacitor de pouca reatância ligado em série
com o enrolamento de partida do estator a fim de proporcionar uma
variação de fase de aproximadamento 90 graus para a corrente de partida,
resultando num torque de partida muito superior ao obtido no sistema
normal de fase divida. 
O capacitor e o enrolamento de partida é desligado por meio de um 
interruptor centrífugo, como se faz no caso normal do motor de fase divida.

%%%%%%%%%%%%%%%%%%%%%%%%%%%%%%%%%%%%%%%%
% pratica, parte II

\section{Prática, parte II}

A segunda parte da experiência consistem em encontrar
os valores práticos para o motor monofásico de partida a capacitor.

\subsection{Corrente de partida}

Utilizando o módulo EMS de fase divida com partida a capacitor
e uma alimentação fixa de 120 volts. A corrente de partida
ficou em 10.6 Amperes para o enrolamento primário. Alimentando
agora, apenas, o enrolamento auxiliar em série com
o capacitor a corrente foi de 7.8 Amperes.

Conectando em paralelo as duas bobinas do motor, principal e 
auxiliar, usando o eletro dinamómetro com sua carga máxima. A
corrente mensurada foi de 11.4 Amperes.

\subsection{Operação com carga variável}

O teste agora foi para um acoplamento com o eletro dinamómetro.
Observação importante, o motor foi alimentando com uma tensão 
de 120 Volts, entretanto devido a grande corrente de consumo
em seus terminais foram obtidos apenas 103.1 Volts.

\begin{center}
    \begin{tabular}{c|c|c|c|c}
            Conjungado (lbf.in) & I (Amperes) & VA & P (Watts) & Velocidade (rpm) \\
            \hline
            0 & 3.00 & 309.30 & 69.00 & 1781 \\
            3 & 3.17 & 326.83 & 123.00 & 1751 \\
            6 & 3.72 & 383.54 & 210.00 & 1717 \\
            9 & 4.513 & 465.30 & 294.00 & 1661 \\
    \end{tabular}
    \captionof{table}{}
\end{center}

O conjugado de partida foi da ordem de 1.3 Newton por metro.

\section{Conclusão}

O Motor monofásico apresenta características bem semelhantes
ao motores de fase divida. Só que o conjugado de partida é levemente
maior que os motores de fase divida. 


\end{document}
