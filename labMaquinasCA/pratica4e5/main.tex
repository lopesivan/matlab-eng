%\documentclass[a4paper, 10pt]{article}
\documentclass[paper=a4, fontsize=11pt]{article}


\usepackage[brazil]{babel}
\usepackage[utf8]{inputenc}
\usepackage{listings}
\usepackage{color}
\usepackage{amsthm}
\usepackage{graphicx}
\usepackage{cite}

\usepackage{schemabloc}
\usetikzlibrary{circuits}

\usepackage{tabularx,ragged2e,booktabs,caption}

\usepackage{hyperref}

\setlength{\parindent}{0pt}
\setlength{\parskip}{18pt}

\title{Laboratório de Máquinas de Corrente Alternada\\Motor Monofásico de Fase Dividida, Parte I e II}
\author{Felipe Bandeira da Silva\\1020942-X}
%\date{}

\begin{document}


\maketitle

%\newpage

%\begin{abstract}
\textit{Este laboratório tem como objetivo: Conhecer as conexões básicas do motor.
Observar as operações de partida e de marcha do motor monofásica de fase divida.
Medir as características de partida e funcionamento do motor monofásico de fase
divida, em condições de carga e vazio. Estudar o fator de potência e a eficiência 
do motor monofásico de fase divida.}
%\end{abstract}

\newpage

\tableofcontents

\newpage

%\listoffigures


%%%%%%%%%%%%%%%%%%%%%%%%%%%%%%%%%%%%%%%%%%%%%%%%%%%%%%%%%%%%%%%%%%%%%%%%%%%%%%%%
% fundamentação teórica
\newpage
\section{Introdução}

Os motores monofásicos apresentam bastante semelhança com os motores
de gaiola de esquilo. São diferentes pelas partes em relação a 
disposição dos enrolamentos do estador. No lugar de uma bobina
concentrada, o enrolamento real do estator está distribuído em 
ranhuras de modo a produzir uma distribuição espacial de FMM 
aproximadamente senoidal. Quando está em repouso, é evidente que
por simetria esse motor basicamente não apresenta nenhum conjugado
de partida porque ele está produzindo conjugados iguais em ambos
os sentidos. Para que este motor tenha partida, ou rotação no seu
eixo é necessário uma movimento inicial. Movimento este produzido
por uma bobina auxiliar. Controlada por um sistema eletro mecânico
que faz a mesma entrar e sair em momentos específicos e dependentes
do estado do motor em rotação e consumo de corrente.

Os motores de indução monofásicos são classificados de acordo 
com os seus métodos de partida. O método estudado na experiência
é o de fase divida. 

Os motores de fase divida apresentam dois enrolamentos no estator, 
o enrolamento principal (também referido como enrolamento de trabalho)
que será indicado pelo subscrito ``principal'' e o enrolamento auxiliar.
Como em um motor bifásico, os eixos desses enrolamentos estão deslocados
entre si de noventa graus elétricos no espaço. O enrolamento auxiliar
tem uma razão mais elevada entre resistências e reatância do que 
o enrolamento principal. A consequência disso é que as duas correntes
estarão fora de fase. Os motores de fase divida têm conjugados de partida
moderados para uma baixa corrente de partida. Aplicações típicas incluem
ventiladores, sopradores, bombas centrífugas e equipamento de escritório. 
As potências nominais tipicas estão entre 50 e 500 watts. Dentro desta
faixa, são os motores de menor custo disponíveis.

Motores de fase divida apresentam uma alta corrente de partida. 
Compreendendo valores de 4 a 5 vezes a corrente nominal em plana
carga. Produzindo dois efeitos, o primeiro é o aumento da temperatura
do motor durante a partida. A segunda, a já falada, corrente de partida
que prova uma grande queda de tensão, reduzindo assim o conjugado
de partida. 

Correntes a vazio produzidas pelo motor de fase divida é 
da ordem de 60 a 80 porcento de plena carga. Valores estes que 
são elevados quando comparados com motores gaiola de esquilo. São
motores barulhentos que os equivalentes trifásicos, devido
a sua vibração mecânica.

%%%%%%%%%%%%%%%%%%%%%%%%%%%%%%
% parte prática
\section{Prática, parte I}

\begin{enumerate}
    \item O enrolamento principal do motor de fase divida está submetido 
            a uma tensão de 100 Volts. 
            \begin{enumerate}
                \item O motor produz um ruído branco.
                \item O motor não girou.
            \end{enumerate}

    \item O eixo agora é submetido a uma força externa, mão do aluno.
            \begin{enumerate}
                    \item O motor não girou.
                    \item O sentido da rotação foi determinado pelo aluno com a mão.
            \end{enumerate}

    \item Agora apenas o enrolamento auxiliar esta submetido a tensão de 100 volts.
            \begin{enumerate}
        \item O motor produziu um ruído branco.
        \item O motor não girou.
            \end{enumerate}

    \item Os enrolamentos estão agora em paralelo.
            \begin{enumerate}
                    \item O motor ``deu'' partida.
                    \item O motor produziu um ruído tipico quando a força mecânica.
                    \item Sentido horário.
            \end{enumerate}

    \item Invertendo os cabos de alimentação.
        \begin{enumerate}
            \item O sentido de rotação não mudou.
        \end{enumerate}

\item Usando agora a chave centrifuga e um capacitor.

        \begin{enumerate}
            \item O motor deu partida.
            \item O interruptor centrífugo funcionou.
            \item O tempo de partida foi de aproximadamente 1.5 segundos.
            \item O tacómetro marcou 1790 rpms.
            \item A redução da tensão não alterou a velocidade.
            

        \end{enumerate}

\item Usando a chave centrifuga para a criação de uma histerese no ligamento
        e desligamento do motor.

        \begin{enumerate}
            \item Fluirá corrente em ambos os enrolamentos.
            \item Será produzido um conjugado de partida.
            \item O motor começará a funcionar.
            \item O motor irá funcionar e depois que alcançar uma determinada velocidade a chave centrifuga ira abrir e com isso, desligando o motor e depois ela vai ligar novamente iniciando o ciclo. De ligar e desligar o motor.
        \end{enumerate}

\end{enumerate}

%%%%%%%%%%%%%%%%%%%%%%%%%%%%%%%%%%%%%%%%
% pratica, parte II

\section{Prática, parte II}

A segunda parte da experiência consistem em encontrar
os valores práticos para o motor de fase divida.

\subsection{Corrente de partida}

Utilizando o módulo EMS de fase divida com partida a capacitor
e uma alimentação fixa de 120 volts. A corrente de partida
ficou em 11.13 Amperes para o enrolamento primário. Alimentando
agora, apenas, o enrolamento auxiliar a corrente foi de 10 Amperes.

Conectando em paralelo as duas bobinas do motor, principal e 
auxiliar, usando o eletro dinamómetro com sua carga máxima. A
corrente mensurada foi de 16.2 Amperes.

\subsection{Operação em vazio}

Para um operação em vazio do motor foram obtido os seguintes 
valores, 

\begin{center}
    \begin{tabular}{c|c|c|c|c}
            E (Volts) & I (Amperes) & P (Watts) & Velocidade (rpm) & Vibração \\
            \hline
            120 & 3.08 & 12 & 1783 & Sim \\
            90 & 2.53 & 61 & 1776 & Sim \\
            60 & 1.857 & 53 & 17737 & Sim \\
    \end{tabular}
    \captionof{table}{}
\end{center}

Os valores para a tensão de 30 Volts não foram obtidos porque a corrente
na bobina ultrapassou o valor nominal.

\subsection{Operação em plena carga}

O teste agora foi para um acoplamento com o eletro dinamómetro.

\begin{center}
    \begin{tabular}{c|c|c|c|c}
            Conjungado (lbf.in) & I (Amperes) & VA & P (Watts) & Velocidade (rpm) \\
            \hline
            0 & 3.41 & 405.7 & 123 & 1783 \\
            3 & 3.56 & 422 & 198 & 1768 \\
            6 & 3.90 & 456 & 271 & 1752 \\
            9 & 4.418 & 520 & 365 & 1709 \\
            12 & 4.93 & 575 & 425 & 1685 \\
    \end{tabular}
    \captionof{table}{}
\end{center}

O conjugado de partida foi da ordem de 1.4 Newton por metro.

\section{Conclusão}

A prática mostra que o motor de fase divida requer mais componentes 
e um cuidados a mais da hora de colocá-lo para funcionar. 
É um motor caro e de baixa potência podendo ser utilizado em 
situações simples e onde apenas uma rede monofásica está disponível.

\end{document}
