% Relatório do laboratório 5 de servo
% Felipe Bandeira da Silva
% 27/09/2013

%\documentclass[a4paper, 10pt]{article}
\documentclass[paper=a4, fontsize=11pt]{article}


\usepackage[brazil]{babel}
\usepackage[utf8]{inputenc}
\usepackage{listings}
\usepackage{color}
\usepackage{amsthm}
\usepackage{graphicx}

\usepackage{schemabloc}
\usetikzlibrary{circuits}

\setlength{\parindent}{0pt}
\setlength{\parskip}{18pt}

\title{Lista de Exercícios 01\\Servo 1}
\author{Felipe Bandeira da Silva}
\date{}


\begin{document}


\maketitle


\newpage

\listoffigures

%%%%%%%%%%%%%%%%%%%%%%%%%%%%%%%%%%%%%%%%%%%%%%%%%%%%%%%%%%%%%%%%%%%%%%%%%%%%%%%%
\newpage
\section{Transformada de Laplace}


%%%%%%%%%%%%%%%%%%%%%%%%%%%%%%%%%%%%%%%%%%%%%%%%%%%%%%%%%%%%%%%%%%%%%%%%%%%%%%%%
\newpage
\section{Valor Final}

Determine o valor final da função $f(t)$ cuja transformada de laplace é dada por:
\begin{equation}
    F(s) = \frac{10}{s(s+1)}
\end{equation}
\textbf{Solução 0:}
Expandindo em frações parciais,
\begin{equation}
    F(s) = \frac{10}{s(s+1)} = \frac{10}{s} - \frac{10}{s+1}
\end{equation}
Aplicando a transformada inversa de laplace,
\begin{equation}
    f(t) = 10 u(t) - 10 e^{-t}
\end{equation}
Onde $u(t)$ é a função degrau unitário. Por simples inspeção é possivel
perceber que o valor final da função é 10.

\textbf{Solução 1:}
Aplicando o teorema do valor final de Laplace,
\begin{equation}
    \lim_{s \to 0}{s F(s)} = \lim_{s \to 0}{s \frac{10}{s(s+1)}} = \frac{10}{1} = 10
\end{equation}
Uma solução menos complexa e mais rápida.

%%%%%%%%%%%%%%%%%%%%%%%%%%%%%%%%%%%%%%%%%%%%%%%%%%%%%%%%%%%%%%%%%%%%%%%%%%%%%%%%
\newpage
\section{EDO domínio da frequência}

Represente as seguintes equações diferenciais no domínio do tempo $s=\sigma + j w$:

\subsection{item a}
\begin{equation}
    2 \frac{d^2 x}{d t^2} + 7 \frac{d^2 x}{d t} + 3 = 0
\end{equation}
Aplicando as propriedades de Laplace para equações diferenciais, e considerando
os valores iniciais $x(0)=3$ e $x'(0)=0$,
\begin{equation}
    2(s X(s) - 3 s - 0) + 7(s X(s) - 3) +  3 X(s) = 0
\end{equation}
Desenvolvendo e isolando $X(s)$,
\begin{equation}
    X(s) = \frac{21 + 6 s}{9 s + 3}
\end{equation}

\subsection{item b}
\begin{equation}
   \frac{d^2 x}{d t^2} + 3 \frac{d^2 x}{d t} + 6 x = 0
\end{equation}

\end{document}

