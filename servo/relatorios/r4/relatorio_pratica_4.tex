% Relatório do laboratório 3 de servo
% Felipe Bandeira da Silva
% 09/06/2013

%\documentclass[a4paper, 10pt]{article}
\documentclass[paper=a4, fontsize=11pt]{article}

\usepackage[framed,numbered,autolinebreaks,useliterate]{mcode}

\usepackage[brazil]{babel}
\usepackage[utf8]{inputenc}
\usepackage{listings}
\usepackage{color}
\usepackage{amsthm}
\usepackage{graphicx}

\setlength{\parindent}{0pt}
\setlength{\parskip}{18pt}

\title{Relatório, Laboratório 4.\\Servo 1}
\author{Felipe Bandeira da Silva}


%%%%%%%%%%%%%%%%%%%%%%%%%%%%%%%%%%%%%%%%%%%%%%%%%%%%%%%%%%%%%%%%%%%%%%%%%%%%%%%%
% MAIN
%%%%%%%%%%%%%%%%%%%%%%%%%%%%%%%%%%%%%%%%%%%%%%%%%%%%%%%%%%%%%%%%%%%%%%%%%%%%%%%%

\begin{document}

\maketitle

\begin{abstract}
Utilizar o Matlab para analisar a resposta transitória de sistemas de 1ª ordem
ao degrau e estudar o efeito do controle proporcional sobre os aspectos de
estabilidade, velocidade de resposta e erro em regime permanente.
\end{abstract}


\newpage

%%%%%%%%%%%%%%%%%%%%%%%%%%%%%%%%%%%%%%%%%%%%%%%%%%%%%%%%%%%%%%%%%%%%%%%%%%%%%%%
% QUESTÃO 1
%%%%%%%%%%%%%%%%%%%%%%%%%%%%%%%%%%%%%%%%%%%%%%%%%%%%%%%%%%%%%%%%%%%%%%%%%%%%%%%
\section{Primeira questão}

\begin{equation}
    G_1(s) = \frac{1}{s+1}
\end{equation}

\begin{equation}
    G_2(s) = \frac{1}{s-1}
\end{equation}

\begin{lstlisting}
\end{lstlisting}


\end{document}
