% Felipe Bandeira da Silva
% 28/11/2013

\documentclass[a4paper, 10pt]{article}
%\documentclass[paper=a4, fontsize=11pt]{article}

\usepackage[brazil]{babel}
\usepackage[utf8]{inputenc}
\usepackage{listings}
\usepackage{color}
\usepackage{amsthm}
\usepackage{graphicx}

\usepackage{schemabloc}
\usetikzlibrary{circuits}

\setlength{\parindent}{0pt}
\setlength{\parskip}{18pt}

\title{Lista de Exercício\\Servo 1}
\author{Felipe Bandeira\\Deyd Jackson}
\date{}

\begin{document}

\maketitle

\newpage

\listoffigures

\newpage

1. Esboce o gráfico do lugar das raízes para o sistema abaixo.

A função de transferência em malha aberta é,
\begin{equation}
    G(s) H(s) = \frac{K (s+2) (s+3)}{s(s+1)}
\end{equation}

Esboço do lugar das raízes,




\end{document}
