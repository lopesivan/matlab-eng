% Modelo de Trabalhos em LaTeX da UFC
%
% Na criação deste modelo foi tomada como base os modelos de monografia da UECE
% criados por Rudy Matela e Sergio Correia, sem a ajuda deles este trabalho teria sido muito mais
% difícil. Este modelo utiliza o abnTeX e um pacote (ufc.sty) para formatação
% de alguns anexos necessários da UFC (folha de rosto, CIP, epígrafe, ...).
%
% Este documento não clama possuir conformidade de 100\% com as normas de
% trabalhos da UFC. Consulte os guias oficiais.
%
% Agradeço ao Lincoln que contribui com o brasão da UFC.
% Agradeço ao Regis e a Mônica pelas contribuições com relação ao "Lorem
% ipsum"
%
% OBS: O modelo de monografias da UECE criado por Rudy Matela encontra-se
% disponível em: http://matela.com.br/pub/modelo_monografia/
%
% Autor: Diego Victor Simões de Sousa
% Data: 10/06/2011


\documentclass[pnumabnt,normaltoc,espacoumemeio,capchap]{abnt}		
\usepackage[brazil]{babel}
\usepackage[utf8]{inputenc}
\usepackage{abnt-alf}
\usepackage{graphicx}
\usepackage{ufc}
\usepackage{multicol}
\usepackage{listings}
\usepackage{eufrak}
\usepackage{subfig}
\usepackage[T1]{fontenc}
\bibliographystyle{abnt-alf}
\setcounter{secnumdepth}{3}
\setcounter{tocdepth}{3}

% Informações gerais do documento
\autor{Lourena Maria Domingos}
\autorr{CHAGAS, F. F. S.}
\titulo{}
\local{Fortaleza, Ceará}
\cidade{Fortaleza}
\data{2013}
\orientador{Prof. Dr. Herminio Borges Neto}
\coorientador{Profa. Me. Dina Mara Pinheiro}
\codigocip{A000z}{CDD:000.0}

% Descrição para folha de rosto
\comentario{
Dissertação submetida à Coordenação do Curso
de Pós-Graduação em Ciência da Computação da
Universidade Federal do Ceará, como requisito
parcial para a obtenção do grau de Mestre em
Ciência da Computação.
}

\comentarioaprovacao{
Dissertação submetida à Coordenação do Curso
de Pós-Graduação em Ciência da Computação, da
Universidade Federal do Ceará, como requisito
parcial para a obtenção do grau de Mestre em
Ciência da Computação. 
}

% Informações institucionais
\centro{Centro de Ciências}
\departamento{Departamento de Computação}
\curso{Pedagogia}
\instituicao{Universidade Federal do Ceará}

\tipotrabalho{Dissertação}
%\areaconcentracao{Banco de ...}
%\nivel{Mestrado}
%\tituloacademico{Mestre}

\dedicatoria{Aos meus Pais.}
% Epígrafe: citação e autor
\epigrafe{``Ninguém, ninguém\\Verá o que eu sonhei\\Só você meu amor\\Ninguém verá o sonho\\Que eu sonhei''}
\autorepigrafe{Geraldo Azevedo}

% Membros da comissão avaliadora
\bancaum{\ABNTorientadordata\\Universidade Federal do Ceará - UFC\\Orientador}
\bancadois{Prof. Me. Dina Mara Pinheiro\\Universidade Federal do Ceará - UFC\\Co-orientador}
\bancatres{Prof. Me. Mulher\\Universidade Federal do Ceará - UFC}
%\bancaquatro{Prof. Dr. Zé Ninguém\\Universidade Federal do Ceará - UFC}

% Palavras chave
\pcs{Burocracia}{\LaTeX}{Documentos}
\kws{Bureucracy}{\LaTeX}{Documents}

\begin{document}

\capa
\folhaderosto
%\makecippage
\termodeaprovacao

% Dedicatória (Opicional)
\makededicatoria

\pretextualchapter{Agradecimentos}
Agradeço a Fulano de Tal, Beltrano da Silva e Cicrano das Tantas
pela ajuda na produção do template \LaTeX\ do modelo de trabalhos
científicos da UFC.
\pagebreak

\makeepigrafe
\begin{resumo}
Isto é um template de documento em \LaTeX. Este documento tenta se
aproximar o máximo possível da norma de produção de trabalhos
científicos da UFC. Este documento não possui 100\% de fidelidade
com as normas da UFC. Assim, use-o por sua conta e risco.

\palavraschave
\end{resumo}
\pagebreak

\begin{abstract}
This is a template document in \LaTeX. This document tries to bring as much
of the production standard of UFC scientific work. This document does not have 
100\% fidelity to the rules of the UFC. So, use it at your own risk.

\keywords
\end{abstract}

\listadefiguras
%\listadetabelas
\tableofcontents
% \listadesiglas % \sigla{sigla}{Descrição}
% \listadesimbolos % \simbolo{símbolo}{Descrição}



\chapter{Exemplos de \LaTeX}


\section{Introdução}

Introdução do template. The quick brown fox jumps over the lazy dog. Lorem
ipsum dolor sit amet, consectetur adipisicing elit, sed do eiusmod tempor
incididunt ut labore et dolore magna aliqua. Ut enim ad minim veniam, quis
nostrud exercitation ullamco laboris nisi ut aliquip ex ea commodo consequat.
Duis aute irure dolor in reprehenderit in voluptate velit esse cillum dolore eu
fugiat nulla pariatur. Excepteur sint occaecat cupidatat non proident, sunt in
culpa qui officia deserunt mollit anim id est laborum.

Introdução do template. The quick brown fox jumps over the lazy dog. Lorem
ipsum dolor sit amet, consectetur adipisicing elit, sed do eiusmod tempor
incididunt ut labore et dolore magna aliqua. Ut enim ad minim veniam, quis
nostrud exercitation ullamco laboris nisi ut aliquip ex ea commodo consequat.
Duis aute irure dolor in reprehenderit in voluptate velit esse cillum dolore eu
fugiat nulla pariatur. Excepteur sint occaecat cupidatat non proident, sunt in
culpa qui officia deserunt mollit anim id est laborum.



\section{Exemplos}


\subsection{Descrição}

A descrição é descrita abaixo:

\begin{description}
\item[Aspecto A:] Aspecto A.
\item[Aspecto B:] Aspecto B.
\item[Aspecto C:] Aspecto C.
\end{description}

Lorem ipsum dolor sit amet, consectetur adipiscing elit. Aliquam pellentesque semper urna, vel ullamcorper nibh lobortis in. Duis consequat, tortor sit amet laoreet pharetra, felis mauris gravida odio, et pharetra justo tellus non magna. Ut ac eros a ligula porttitor elementum eu sit amet lorem. Integer lorem lorem, dignissim sed blandit in, vestibulum at leo. Ut dignissim sollicitudin magna, ut consectetur risus condimentum pretium. Aliquam vulputate magna at leo placerat ac consectetur quam molestie. Nam feugiat, diam ut suscipit eleifend, tellus justo volutpat diam, blandit auctor nisi tortor id mi. Curabitur sed luctus eros. Donec hendrerit gravida lectus vitae imperdiet. Proin ipsum ipsum, pulvinar et congue sed, pharetra at tellus. Donec dapibus odio at orci facilisis eu accumsan nunc suscipit. Duis tristique ligula vel mi vehicula tincidunt.

Mauris neque nulla, gravida vitae porta et, placerat non nunc. Nulla odio augue, cursus ut congue eget, feugiat vitae justo. Maecenas vel facilisis tellus. Nunc tincidunt, tellus ac malesuada vehicula, diam tortor fermentum justo, vitae ultrices risus urna eu odio. Nunc feugiat, purus sit amet fringilla sagittis, odio tortor aliquet nisl, ac porttitor eros mauris eu orci. Integer sagittis rhoncus lorem ac viverra. Quisque vulputate, massa eget malesuada pretium, eros nisl tristique massa, at mollis risus turpis eget nibh. Sed scelerisque, nibh vitae suscipit varius, quam dui tincidunt turpis, vitae aliquam purus massa id justo. Curabitur quis erat at nisl varius laoreet. Suspendisse potenti. Cras sollicitudin mauris at nibh congue non sollicitudin nunc porta. Curabitur non nibh tortor. Nullam at felis odio. Phasellus faucibus placerat velit, eget lacinia diam rutrum vitae. Nunc nec risus quam, ac egestas leo. Nulla velit turpis, lacinia a accumsan sit amet, dictum sed ipsum. Aenean commodo pretium purus convallis ultrices. Nulla vel enim augue, vitae dictum erat.

Nullam tempus malesuada mi, ac sodales orci accumsan eget. Aenean mollis odio in lectus pellentesque quis consectetur velit tempor. Lorem ipsum dolor sit amet, consectetur adipiscing elit. Ut sed bibendum arcu. Etiam vitae nunc iaculis risus laoreet commodo. Integer euismod, sem quis sagittis mattis, enim eros laoreet sapien, vitae euismod urna ante nec risus. Nunc pellentesque augue sed nisl tristique semper. Duis id arcu eget dui imperdiet fermentum. Nunc iaculis, eros lobortis vehicula molestie, elit neque rutrum augue, sit amet varius lorem mauris at orci. Integer egestas nisi arcu, a adipiscing lorem.

Sed a tellus in lacus porta rhoncus. Curabitur vehicula ligula quis elit hendrerit at cursus lacus malesuada. Aenean fringilla, nibh a porta volutpat, mi est fermentum tortor, congue malesuada ligula tortor et massa. Sed tempus, nisl a consectetur commodo, urna elit faucibus nulla, vel tempus eros ligula non urna. Sed hendrerit mattis odio sit amet condimentum. Nullam volutpat interdum dapibus. Fusce sed iaculis leo. Quisque blandit nunc leo. In augue ligula, blandit condimentum dignissim eu, dapibus vitae lorem. Nullam blandit, eros id tristique convallis, libero dolor feugiat mi, vitae pretium tortor nisi non est. Nullam ornare ultrices imperdiet. Vivamus eu neque a neque accumsan rhoncus a in justo. In tellus risus, sollicitudin nec ullamcorper quis, faucibus nec libero. Pellentesque non lacus quis magna dapibus bibendum et pulvinar ipsum. In tincidunt, odio non mattis aliquam, orci est dignissim lectus, ac pellentesque nisl tortor sit amet lacus. Aliquam ornare nunc non odio pharetra non condimentum tortor consectetur. Morbi imperdiet feugiat libero, a luctus nisl elementum vitae. Fusce mauris lorem, placerat vitae semper at, varius at purus.

Quisque aliquet porta quam. Nulla aliquet sem ac mauris vestibulum commodo. Sed in ante purus. Integer quis viverra eros. Duis dapibus odio eu risus dignissim pulvinar. Aliquam ut tellus sed tellus consequat consectetur ut sed orci. Phasellus sollicitudin dolor eu dui venenatis vulputate sed vel velit. Proin laoreet ipsum sollicitudin dui porta in ullamcorper purus ornare. Nam sed interdum ipsum. Proin at elit eget arcu feugiat consectetur 

\subsection{Figura}

A figura \ref{fig:graph} mostra uma figura. Quidquid latine dictum sit altum
viditur. The quick brown fox jumps over the lazy dog. Quidquid latine dictum
sit altum viditur. The quick brown fox jumps over the lazy dog.

\begin{figure}[htbp]
\centering
\includegraphics[width=.30\textwidth]{fig/UFC}
\caption{Brasão da UFC}
\label{fig:graph}
\end{figure}


\subsection{Tabela}

A tabela \ref{tab:tabela} mostra uma tabela. Quidquid latine dictum sit altum
viditur. The quick brown fox jumps over the lazy dog. Quidquid latine dictum
sit altum viditur. The quick brown fox jumps over the lazy dog.

\begin{table}[htbp]
	\caption{Tabela}
	\label{tab:tabela}
	\centering
	\begin{tabular}{|c|l|r|}
		\hline
		The 	&	Quick 	&	Brown	\\
		\hline
		Fox	&	Jumps	&	Over	\\
		The	&	Lazy	&	Dog	\\
		\hline 
	\end{tabular}
\end{table} 


\subsection{Citações (Referências)}

De acordo com \cite{DEAD:1666,BEEF:1234} este paragrafo exemplifica referências
(citações). Lorem ipsum dolor sit amet, consectetur adipisicing elit, sed do
eiusmod tempor incididunt ut labore et dolore magna aliqua. Ut enim ad minim
veniam, quis nostrud exercitation ullamco laboris nisi ut aliquip ex ea commodo
consequat. Duis aute irure dolor in reprehenderit in voluptate velit esse
cillum dolore eu fugiat nulla pariatur. Excepteur sint occaecat cupidatat non
proident, sunt in culpa qui officia deserunt mollit anim id est laborum.
Quidquid latine dictum sit altum viditur.


\section{Lorem Ipsum}

Lorem ipsum dolor sit amet, consectetur adipisicing elit, sed do eiusmod tempor
incididunt ut labore et dolore magna aliqua. Ut enim ad minim veniam, quis
nostrud exercitation ullamco laboris nisi ut aliquip ex ea commodo consequat.
Duis aute irure dolor in reprehenderit in voluptate velit esse cillum dolore eu
fugiat nulla pariatur. Excepteur sint occaecat cupidatat non proident, sunt in
culpa qui officia deserunt mollit anim id est laborum.


%\chapter{Lorem ipsum dolor}
\label{cap:fundamentos}

Lorem ipsum dolor sit amet, consectetur adipiscing elit. Nulla bibendum urna at risus rutrum aliquet. Class aptent taciti sociosqu ad litora torquent per conubia nostra, per inceptos himenaeos. Suspendisse faucibus suscipit lectus id ullamcorper. Cum sociis natoque penatibus et magnis dis parturient montes, nascetur ridiculus mus. Quisque quis nulla eget lorem tincidunt congue. Aenean in augue vitae neque mattis tincidunt. Sed ipsum purus, molestie accumsan porttitor non, blandit ac libero. Proin pharetra purus eget mauris interdum sed congue urna scelerisque. Etiam malesuada, ligula id feugiat feugiat, ligula felis sagittis est, sed tincidunt tortor nisl at dolor. Class aptent taciti sociosqu ad litora torquent per conubia nostra, per inceptos himenaeos. Quisque et sapien ligula. In lacinia lobortis ligula ac consequat. Praesent nisi nisl, lacinia sed lobortis ut, eleifend non quam. Sed eget orci nibh. Aenean vel mi nec libero placerat feugiat. Vivamus leo leo, aliquet ut consectetur eget, vehicula ac urna. Sed convallis, tortor nec porttitor blandit, arcu massa imperdiet lectus, et dictum lorem ligula tempus mi. Nunc non dui massa, nec ultricies purus. Curabitur neque metus, luctus nec tempor ut, egestas ac mauris. Etiam eget mi magna, ac facilisis metus.

Aenean et nunc ut risus pharetra tristique a et purus. Mauris nec rutrum tortor. Aliquam bibendum erat urna, nec volutpat lacus. Proin in nunc lacus. Fusce commodo, orci sit amet adipiscing sollicitudin, libero mauris porttitor mi, quis iaculis nibh elit at sem. Ut aliquet nulla et diam tristique quis egestas odio elementum. Aliquam malesuada faucibus velit, ut pellentesque dui rutrum vel. Quisque faucibus condimentum mi nec varius. Nulla facilisi. Quisque ultricies diam at augue pretium ac ultrices metus pretium. Aenean cursus volutpat augue, ac interdum eros accumsan ut. Class aptent taciti sociosqu ad litora torquent per conubia nostra, per inceptos himenaeos. Sed eu erat nunc, vitae rutrum odio. Maecenas lobortis dignissim enim vel commodo. Nam ut porta ipsum. Nunc convallis auctor massa, quis blandit mauris laoreet eget. Suspendisse a nibh commodo leo dapibus pharetra eget id ante. Donec sit amet justo quis urna laoreet convallis.

Duis volutpat euismod turpis, vitae faucibus risus porta ac. Pellentesque fringilla risus ut mauris fringilla consequat. Nunc porta pulvinar est, ac pulvinar mi gravida ut. Quisque ut nulla erat. Aliquam erat volutpat. Aenean volutpat lacus quis mi ullamcorper eget ultrices augue adipiscing. Pellentesque sed mauris mauris, sit amet porta ipsum. Sed nisl neque, dignissim at rhoncus ac, convallis quis nibh. Nulla adipiscing dapibus odio, eu scelerisque urna ultrices in. Class aptent taciti sociosqu ad litora torquent per conubia nostra, per inceptos himenaeos. Etiam fringilla blandit nulla ut mattis. Phasellus nunc massa, mollis eu venenatis et, porta a tellus. Nam porta, metus at lobortis tempus, augue velit porttitor arcu, cursus mollis nisi mauris at lectus. Suspendisse quis neque mi, faucibus faucibus purus.

\section{Aenean et nunc}
\label{subsec:tbloc}

Aenean et nunc ut risus pharetra tristique a et purus. Mauris nec rutrum tortor. Aliquam bibendum erat urna, nec volutpat lacus. Proin in nunc lacus. Fusce commodo, orci sit amet adipiscing sollicitudin, libero mauris porttitor mi, quis iaculis nibh elit at sem. Ut aliquet nulla et diam tristique quis egestas odio elementum. Aliquam malesuada faucibus velit, ut pellentesque dui rutrum vel. Quisque faucibus condimentum mi nec varius. Nulla facilisi. Quisque ultricies diam at augue pretium ac ultrices metus pretium. Aenean cursus volutpat augue, ac interdum eros accumsan ut. Class aptent taciti sociosqu ad litora torquent per conubia nostra, per inceptos himenaeos. Sed eu erat nunc, vitae rutrum odio. Maecenas lobortis dignissim enim vel commodo. Nam ut porta ipsum. Nunc convallis auctor massa, quis blandit mauris laoreet eget. Suspendisse a nibh commodo leo dapibus pharetra eget id ante. Donec sit amet justo quis urna laoreet convallis.

\subsection{Blih blih blih} 
 
Sed quis euismod massa. Nunc et ultricies nisi. Maecenas feugiat magna eu nibh feugiat gravida. Cras nec leo tincidunt augue porttitor consequat. Vestibulum scelerisque semper augue, tempor ultricies velit pharetra ac. Pellentesque at nunc felis, vitae consectetur dolor. Fusce at turpis augue. Nunc nec lectus odio. Vestibulum non libero purus. In sed enim elit, ut pretium sapien. Curabitur eros risus, pharetra vel vestibulum sed, aliquam ut lacus.


\section{Nullam id tellus}
\label{sec:lateracao}
 
Nullam id tellus velit. Sed vitae sollicitudin mi. Aliquam felis sapien, ullamcorper quis adipiscing at, mollis sed eros. Class aptent taciti sociosqu ad litora torquent per conubia nostra, per inceptos himenaeos. Suspendisse cursus leo at neque egestas in facilisis dui molestie. Proin tincidunt tellus sed enim vulputate aliquam. Morbi placerat nunc eget mi scelerisque in euismod neque accumsan. Aliquam vestibulum volutpat lacus, in dictum quam molestie a. Suspendisse massa lectus, posuere eget tincidunt sed, adipiscing at justo. Nunc scelerisque, elit non posuere semper, lorem magna lacinia justo, et ullamcorper arcu diam non nunc. Pellentesque habitant morbi tristique senectus et netus et malesuada fames ac turpis egestas. Cras vel metus arcu, non venenatis magna. Phasellus viverra lacinia sem ac sagittis. Nullam vel mi nunc, eu congue diam. Praesent rutrum ligula eget tortor posuere ornare. Nullam tempor sem vitae ipsum venenatis eget eleifend libero ultricies. In sagittis euismod aliquam. Cum sociis natoque penatibus et magnis dis parturient montes, nascetur ridiculus mus. Maecenas convallis congue vehicula. Nunc interdum commodo suscipit.



\bibliography{bib}

%\appendix

%\include{cha}

\end{document}

