\chapter{Chegada do Formador}

Neste capítulo, apresentaremos uma breve descrição da disciplina, mostrando elementos que caracterizam o formador\textbackslash{}tutor quando este inicia seu trabalho no MM, as atividades desenvolvidas pela equipe de bolsistas, assim como alguns conceitos básicos sobre a temática abordada que é fundamental na compressão do objeto deste estudo, sob a ótica de Young, Borges Neto, Almeida e Perrenoud.

\section{Descrição da Disciplina Educação a Distância}

A Disciplina de Educação a Distância, consta na grade curricular do curso de pedagogia presencial desde o ano de 2003, inicialmente como disciplina optativa passando a ser obrigatória a partir da reforma curricular seguinte. Vinculada ao departamento de estudos especializados, semestralmente é ofertada 4 turmas aos alunos do terceiro período do curso e tem sua carga horaria desenvolvida 75$\%$ a distância e 25$\%$ presencial, assegurada pela portaria do MEC 2.253 (18\textbackslash{}10\textbackslash{}2008), que registra que os cursos de ensino superior federais devem contemplar 20$\%$ de sua oferta com disciplinas que acontecem de forma não presencial.

Conforme os Referencias de Qualidade para a Educação Superior a Distância (MEC, 2007), ``não há um modelo único de EaD. Os programas podem apresentar diferentes desenhos e múltiplas combinações de linguagens e recursos educacionais e tecnológicos'' (p.7).  Desta forma, esse modelo consta de uma equipe composta de dois professores titulares e oito formadores (tutores) estudantes do curso de graduação, bolsistas do MM, conforme Young:

\begin{citacao}
Em outras universidades ou instituições que oferecem educação a distância, é comum a utilização do termo ''tutor'', para denominar os sujeitos que atuam na educação virtual, auxiliando o professor titular; no entanto, essa nomenclatura traz a ideia do profissional que vai organizar a pratica educativa em temos de controlo, e, em sua concepção mais tradicional, é aquele que ''cuida'' do aluno. Sua função é verificar se os alunos estão entregando as atividades nos prazos, se estão entrando no ambiente virtual de educação, entre outras atividades dessa natureza. Esses acompanhamentos são necessários, mas a função dos formadores não se restringem a isso.(p.69)
\end{citacao}

Esta equipe é responsável por toda a execução da disciplina que se desenvolve nos ambientes virtuais \textit{Moodle} e \textit{Teleduc}. Os dois constituem- se de softwares livres, sendo o primeiro um software australiano utilizado pelo mundo inteiro desde 1999 e o segundo desenvolvido pela Universidade Estadual de Campinas, UNICAMP desde 1998, além de atividades realizadas na rede social \textit{Facebook}, através da ferramenta grupos, está que foi criada nos Estados Unidos da América em 2004.

A ideia de utilizar mais de um ambiente é possibilitar ao discente que ele conheça mais de uma plataforma virtual e suas possibilidades, já que esta disciplina é a única que permite a experiência virtual.

Os ambientes virtuais \textit{Moodle} e \textit{Teleduc} estão hospedados no site do laboratório, www.multimeios.ufc.br e clicar no ícone ``virtual multimeios'' conforme indica a imagem abaixo:

\begin{figure}[!ht]
    \centering
    \includegraphics[scale=.5]{f1.png}
    \caption{Caminho para chegar aos ambientes virtuais}
    \label{fig:caminho}
\end{figure}

Esse caminho permite acesso à página de projetos do MM, incluindo os ambientes virtuais citados, que para ter acesso é só clicar no ícone referente a cada um deles:

\begin{figure}[!ht]
    \centering
    \includegraphics[scale=.5]{f2.png}
    \caption{Página de acesso aos AV’s}
    \label{fig:acessoav}
\end{figure}

Já para entrar no grupo da rede social, os alunos devem fazer login em suas contas pessoais no \textit{Facebook} acessando \underline{www.facebook.com.br} e procurar pelo grupo da disciplina que é sempre divulgado no primeiro dia de aula. Após a solicitação é só aguardar a aprovação de um dos administradores para ter acesso às informações e atividades.

\section{Perfil do Formador}

Pensando em traçar o perfil do formador, apresentaremos quem é esse sujeito e como ele chega a esta condição. 

Para ser formador da disciplina é necessário participar e um processo seletivo que acontece no inicio do ano com a oferta de bolsas para as disciplinas obrigatórias. A seleção consta de duas etapas, sendo a primeira uma prova escrita, seguida de uma entrevista com os professores da disciplina e demais membros do MM. Nesse processo são selecionados dois bolsistas, um remunerado e outro voluntário que serão vinculados ao projeto de monitoria intitulado Formação de mediadores na modalidade à distância. Para participar da seleção é necessário ter cursado a disciplina anteriormente que é, para muitos, o primeiro contato com o ensino a distância, onde é proporcionado a ele a vivência nesta modalidade, então todos os bolsistas ao começar as atividades como formador já trazem o conhecimento prévio sobre a modalidade aprendida enquanto estudante da disciplina.  Os demais formadores são bolsistas do laboratório vinculados a outros projetos que atuam como formadores voluntários com critério também de terem cursado a disciplina anteriormente. Como há uma rotatividade, tendo em vista o processo seletivo, a equipe está sempre se renovando. 

Podemos considerar que neste primeiro momento o bolsista não tem formação para exercer atividades como docentes, e este logo quando chega já vai atuar como formador desenvolvendo atividades junto com os professores, portanto é a partir dessa chegada que começa a formação dos formadores, partindo do principio da ``formação em serviço'', que considera que ao mesmo tempo em que aprende o sujeito também executa, forma enquanto é formado, de forma que assim ele tem um aprendizado contextualizado, que segundo Borges Neto:

\begin{citacao}
Contextualizar conteúdos é reconhecer em primeiro lugar a importância do cotidiano dos/as estudantes no processo educativo e mostrar e demonstrar que os conhecimentos gerados nesse processo de ensino-aprendizagem podem ter aplicação prática na vida das pessoas, de forma geral. Significa compartilhar elementos para que os/as estudantes apreendam o saber, não como armazenamento de conhecimentos técnico-científicos, mas como potencial para enfrentar o mundo de significações e em suas significações.  Borges Neto (2012 p.8)
\end{citacao}

\section{Como funciona a disciplina}

A disciplina é organizada por etapas que começa logo após a seleção dos bolsistas, antes de começar as aulas do semestre, iniciando com a escolha dos materiais e das atividades da disciplina. Nesse momento, toda a equipe se encontra pelo menos duas vezes por semana nas salas anexo ao MM, localizada no prédio NUPER para pensar, organizar e preparar as atividades que serão desenvolvidas ao longo do semestre.

Como não é proposta do laboratório trabalhar com produção de material para a disciplina, durante as primeiras reuniões é solicitado aos formadores, sugestões de textos que possam ser utilizados como referencial teórico no semestre, o material coletado é analisado pela equipe e selecionado. Após a escolha é feito os grupos de estudos onde cada formador apresenta um texto para discutir com o grupo e neste momento propor uma atividade para esse texto e em qual ambiente ela deve acontecer, esta, pode vir a serem umas das atividades do semestre ou se o grupo não concordar, é sugerida outra atividade que se adeque. Desta forma todo o material utilizado no semestre é estudado antes de começar as aulas. Nestas reuniões é produzido o cronograma de atividades, onde se encontra organizado a atividade proposta, o texto base, o prazo para realização e o ambiente aonde irá se desenvolver. Com o cronograma pronto começa a etapa de preparação dos ambientes virtuais. Conforme Almeida 2003, estes:

\begin{citacao}
Permitem integrar múltiplas mídias, linguagens e recursos, apresentar informações de maneira organizada, desenvolver interações entre pessoas e objetos de conhecimento, elaborar e socializar produções tendo em vista atingir determinados objetivos. Almeida (2003, p.331)
\end{citacao}

Para começar a preparação o administrador abre as turmas virtuais, uma em cada ambiente, e cadastra os bolsistas com perfil de formador que confere a eles a possibilidade de manusear como administradores, podendo editar todas as ferramentas, cadastrar os alunos, inserir materiais, preparar e abrir as atividades, entre outros. Com esse perfil o formador tem as mesmas permissões que os professores, permitindo total autonomia com a turma. A equipe se divide e é atribuída funções a cada formador, que pode ser individual ou em duplas, de forma que todas as ferramentas utilizadas sejam contempladas.

É nesse período também que é elaborada a tabela de avaliação com a nota atribuída a cada atividade e os critérios de avaliação. 

\begin{citacao}
A avaliação deve, então, servir de orientação para que o professor possa realizar os ajustes necessários ao seu fazer didático de maneira a transformar as dificuldades em momentos de aprendizagem para seus alunos. Nessa perspectiva, a avaliação torna-se um ``instrumento privilegiado de uma regulação contínua das diversas intervenções e das situações didáticas'' (PERRENOUD, 1999,p.14). 
\end{citacao}

A avaliação perpassa pelas três formas de avaliar em EAD, sendo diagnóstica, formativa e somativa. Diagnóstica, pois no momento em que o aluno chega à disciplina é questionado que conhecimentos ele tem acerca da modalidade, formativa, pois compreende o processo de aprendizagem durante o semestre, considerando os erros do aluno como oportunidade para que ele possa reconstruir seus conceitos, e somativa por que ao final é atribuída nota as atividades que foram produzidas, a fim de obter aprovação curricular.

O contato inicial com os alunos acontece no primeiro dia de aula onde é realizado o primeiro encontro presencial que é apresentado tudo que se refere à disciplina, desde a equipe a atividades que irão ser realizadas e a forma que serão avaliados. Neste momento os alunos realizam seus cadastros nos AV’s e solicitam a inclusão no grupo da rede social. Essas são as primeiras ações que o formador acompanha diretamente os alunos. 

Com os cadastros realizados, os alunos são distribuídos aleatoriamente em grupos, que servem apenas para um acompanhamento mais sistematizado, já que cada grupo é acompanhado por dois formadores. A escolha desta dupla é feita sem tomar como base nenhum critério especifico, só leva em consideração que seja um formador veterano e um formador novato.

Com o inicio do período a equipe continua se encontrando semanalmente para dar continuidade às ações desenvolvidas, que deixa de ser de preparação e passa a ser de execução da disciplina, como ajustes no ambiente virtual, atualização das agendas de acordo com a atividade do momento, se o aluno não conseguir se cadastrar sozinho o formador faz seu cadastro manual, além de discutir e avaliar o andamento da disciplina e o desenvolvimento dos discentes. Há também uma escala de revezamento para atendimento online que acontece através da ferramenta MM Online, que compreende os dias uteis da semana nos turnos tarde e noite. Esta é desenvolvida por Crafty Syntax Live Help (2003 –2009, Eric Gerdes) e adaptada por BORGES NETO, H e BORGES, Daniel Capelo, está alocada no site do Laboratório desde 2006. É semelhante a outras ferramentas de bate-papo, possuindo um layout fácil e com versões disponíveis em diferentes linguagens. 

\section{Percurso de formação do bolsista}

Como dissemos anteriormente essa investigação parte da concepção de formação em serviço, desta forma enquanto atua desenvolvendo as atividades da disciplina, o bolsista formador está sendo formado, desenvolvendo um conjunto de saberes e práticas para atuar com a docência online. Essa formação se fundamenta na proposta teórico metodológica Sequencia Fedathi, que foi desenvolvida pelo grupo FEDATHI na década de 90 e se baseia na aprendizagem por resolução de problemas, levando em consideração a aprendizagem significativa por parte do aluno. É desenvolvida em quatro etapas, tomada de posição, maturação, solução e prova, onde em cada uma delas o professor aparece como mediador na construção do conhecimento por parte do aluno. 

\begin{citacao}
A Sequencia Fedathi, propõe possibilitar ao aluno a elaboração significativa de conceitos, mediante a solução de problemas, cuja as produções serão o objeto sobre o qual o professor vai conduzir a mediação, a fim de levá-lo a construir o conhecimento. Souza (2013 p. 18) 
\end{citacao}

Durante o percurso o formador vivência diversas situações problemas em que os professores participam mediando, sem dar um resposta direta, mas sim levando o formador a pensar na melhor forma de solucionar o problema. Tendo assimilado esse conceito e essa metodologia o formador passa a solucionar sozinho os problemas com os alunos.


