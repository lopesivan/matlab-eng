% Felipe Bandeira
%\documentclass[a4paper, 10pt]{article}
\documentclass[paper=a4, fontsize=11pt]{article}

\usepackage[brazil]{babel}
\usepackage[utf8]{inputenc}
\usepackage{listings}
\usepackage{color}
\usepackage{amsthm}
\usepackage{graphicx}

%\usepackage{schemabloc}
%\usetikzlibrary{circuits}

\setlength{\parindent}{0pt}
\setlength{\parskip}{18pt}

\title{MUMPS}
\author{Felipe Bandeira da Silva}

\begin{document}

\maketitle

\begin{center}
Documetação do trabalho feito na biblioteca MUMPS
\end{center}

\newpage
\listoffigures

\newpage
\tableofcontents

%%%%%%%%%%%%%%%%%%%%%%%%%%%%%%%%%%%%%%%%%%%%%%%%%%%%%%%%%%%%%%%%%%%%%%%%%%%%%%%%

\newpage

\section{Introdução}

É muito mais simples construir um diário de desenvolvimento. E por esta razão
esse documento não é capaz e nunca será capaz de retirar qualquer duvida que
irão surgir no decorrer da leitura. A única e exclusiva intenção, é registrar
os fatos ou eventos que ocorrem no desenvolvimento e estudo da ferramenta
MUMPS.

\section{Construindo o MUMPS}

Para a construção da biblioteca MUMPS se faz necessário o domínio de algumas
ferramentas de desenvolvimento de software do projeto GNU. Entre as inúmeras
as que são de fato mais visíveis e de grande importância são:

\begin{itemize}
	\item gcc
	\item make
\end{itemize}

Boa parte do MUMPS é feita em Fortran, para isso é necessário a instalação 
do compilador. Em muitas distribuições \textit{debian} o apt-get já é de 
grande utilidade. Sugiro o \textbf{gfotran}. Que até o momento não apresentou
qualquer erro. Vale ressaltar que o gfortran tem como base o Fortran 90 e não
o Fortran 77 base do MUMPS, tal situação até o momento não gerou falhas graves
no processamento dos cálculos pelo MUMPS.
Para a execução em paralelo o MUMPS requer um gerenciador
de processos, filas, recursos... Para tanto foi escolhido o MPI. Escolha
feita para a compatibilidade os clusters do CENAPAD UFC. 

\subsection{Biblioteca BLAS}

A maior fonte de dores de cabeça é sem sombra de duvidas a biblioteca BLAS. 
Sua compilação requer inúmeras mudanças do PATH do Makefile e permutações de
compiladores. Esta biblioteca pode ser facilmente obtida na \textit{netlib}.
Partindo do ponto em que as ferramentas básicas de compilação já estão
devidamente instaladas na maquina alvo. No meu caso um Linux/Mint 13 Maya.

Importante lembrar que o projeto BLAS é datado nos anos 90 e com ultima 
atualização importante em 97. A novas atualização são para manutenção da biblioteca
em novas arquiteturas, em sua maioria computadores de alto poder de processamento.






\end{document}

