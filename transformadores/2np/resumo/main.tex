%\documentclass[a4paper, 10pt]{article}
\documentclass[paper=a4, fontsize=11pt]{article}
\usepackage[brazil]{babel}
\usepackage[utf8]{inputenc}
\usepackage{listings}
\usepackage{color}
\usepackage{amsthm}
\usepackage{graphicx}
\usepackage{schemabloc}
\usetikzlibrary{circuits}
\usepackage{tabularx,ragged2e,booktabs,caption}

\setlength{\parindent}{0pt}
\setlength{\parskip}{18pt}

\title{\textsc{Detecção de componente continua em transformadores de potência}}
\author{Felipe Bandeira da Silva, 1020942-X\\Renan Teixeira, 000}
%\date{}

\begin{document}


\maketitle

%\begin{abstract}
\textit{A presença de cargas não lineares, sejam, inversores de frequência, grandes computadores. Introduzem na rede uma distorção da senoide produzida pela geradores de energia. Essa componente DC injetada na alimentação de transformadores de potência, satura o núcleo do transformador, aumenta a temperatura interna do mesmo. Danificando em muitos casos a isolação. Esse resumo do artigo tem como principal objetivo: detectar a saturação magnetica por um metodo não diretor de medição de corrente continua. E outras palavras a somatória das componentes DC dos equipamento conectados no transformador. A sensor magnético desenvolido pelos autores do trabalho, verifica em um loop fechado e com faixa de rejeição os campos magnéticos, com uma boa linearidade. E no final são confrontados os resultados das simulações e experimental confirmando uma otima aproximação em ambos.}
%\end{abstract}

\newpage

\tableofcontents

\newpage

\listoffigures


%%%%%%%%%%%%%%%%%%%%%%%%%%%%%%%%%%%%%%%%%%%%%%%%%%%%%%%%%%%%%%%%%%%%%%%%%%%%%%%%
% fundamentação teórica
\newpage
\section{Introdução}

O aumento do uso de conversores de potência na industria, é razão, em boa parte das 
situações os problemas referidos na qualidade de energia. Um dos maiores motivos 
para as distorções são as cargas não lineares produzidas pelos conversores. Por essa
razão é e são utilizadas diversos filtros ativos ou passivos na tentativa de melhorar
a qualidade da energia. O problema da componente continua na energia elétrica e 
valores tolerados, varia de pais ou região, os limites são regulados pelas normas 
locais. Entretanto o seguimento das normas não é garantia de correção para esse problema.
É relativamente impossível abrir mão dos conversores de potência nos dias atuais, 
então deve-se trabalhar em cima das harmônicas produzidas e uma forma de correção.
No artigo aqui apresentado o alvo para a analise dos problemas causados pela componente
DC são os transformadores de potência. O principal problema a injeção de corrente 
continua em transformadores é a saturação assimétrica do núcleo magnético durante 
meio ciclo. Quando essa condição é presente o consumo de energia reativa cresce, 
implicando em perda de potência e consequentemente no aumento da temperatura do
transformador. Os autores deste artigo propõem um método indireto e on-line 
de detecção da componente continua injetada no transformador. A literatura
apresenta diversas  alternativas on-line e offline no diagnostico na detecção 
de falhas e monitoramento. Existem métodos que verificam instantaneamente 
curtos em vários estágios da transmissão, mensuramento apenas o fluxo de 
corrente do transformador. Outra técnica é baseada na analise da função de transferência
do circuito, quando injetando um sinal conhecido, verifica-se tensões internas
nos taps do transformado tensões externa, fazendo isso a diferenciação entre
erros internos do transformado e erros externos. Em particular a medição de componente
continua injetada em sistemas de potência, não são muito exploradas na literatura.

A principal ideia e proposta do artigo é quando a componente continua flui pelo
transformador existem quedas de tensão na resistências parasitas dos enrolamentos.
Essa queda de tensão continua, guarda informações valiosas sobre a saturação
no núcleo do transformador. Essa queda de tensão entretanto é muito pequena (para um
transformador de potência a resistência dos enrolamentos é da ordem de miliohns), 
sensores tradicionais não são capazes de detectar a componente continua, já que 
são exclusivamente voltados para a medição AC. Entretanto a utilização dos sensores
de efeito Hall apresenta pequenos problemas, existem uma grande dificuldade de 
separar a componente DC muito pequena das componentes AC grandes. Os autores do 
artigo criaram um sensor magnético não linear com uma grande sensibilidade para
componente DC. Mensurando assim, as pequenas componentes DC encontrada na resistências
parasitas dos enrolamentos do transformador. Os autores apresentam duas diferentes
implementações para  a medição. Uma é em malha aberta e outra em malha fechada. 
A primeira opera na medição do fluxo de DC dentro do sensor do núcleo(em malha
fechada é visto o fluxo de harmônicas). A simulação e os resultados experimentais
confirmam a eficiência do método. Em particular os ótimos resultados adquiridos no 
métodos da malha fechada.

\section{Principio de operação}

O elemento chave proposto é um sensor com alta precisão para mediação de corrente
continua com uma razão de rejeição em modo comum alta. Lembrando que o sensor 
é capaz de extrair uma componente continua na ordem de milivolts de um sinal
com tensão de pico a pico na ordem de 600 $V_{rms}$. O sensor feito pelos autores
é composto por um núcleo toroidal, com dois enrolamentos e um reator para a compensação.
A componente magnética é quantificada, imitando a baixa potência do transformador
toroidal com a tensão primária. O secundário é então utilizado para garantir o 
loop fechado. O modelo para o componente magnético do sensor, chamado pelos autores
de reator simples, é apresentado na Figura~\ref{fig:figura1}.

\begin{figure}[!ht]
	\centering
	\includegraphics[scale=.8]{fig1.png}
    \caption{Esquemático do reator, conectado a fonte AC com a componente DC}
    \label{fig:figura1}
\end{figure}

A figura~\ref{fig:figura1} mostra um resistência em séria uma indutância não linear
o enrolamento secundário não é considerado ainda, o reator é então conectado a
uma fonte de corrente AC com uma componente DC. A corrente $i_R$ do reator, é
detectada por um transformador de corrente. O transformador de corrente
é implementado para a correção do offset na medição. Para uma baixa qualidade na 
medição um simples resistor shunt pode ser utilizado.
A operação do sensor é atuante na saturação assimétrica apresentando assim a componente
DC no funcionamento em regime permanente do transformador. A figura ~\ref{fig:figura2} 
mostra as formas de onda para o fluxo magnético no reator, corrente de magnetização
quando uma componente continua esta presente na rede. Nota-se que o fluxo magnético é
mais intenso no fim do semiciclo da senoide.

\begin{figure}[!ht]
	\centering
	\includegraphics[scale=.8]{fig2.png}
    \caption{Tensão do reator(azul), fluxo magnético(verde) e corrente de magnetização(vermelho), na presença da componente DC.}
    \label{fig:figura2}
\end{figure}

O resultado apresenta que, quando a corrente do reator presente for alta é correspondente
ao semi ciclo positivo ou negativo dependente da componente DC. A saturação assimétrica
do núcleo é informada pela componente DC em amplitude. Para detectar a distorção
assimétrica em todo os ciclos da senoide é feita uma integração para a saturação positiva
e negativa em um ponto bem próximo a transição do zero.

No caso da tensão DC seja positiva equivale a semi ciclo positivo do fluxo de saturação e
a distorção assimétrica, equivalente as harmônicas pares. No caso do modulo da integral
para o semi ciclo positivo seja maior que o modulo da integral para o semi ciclo negativo
a valor e sinal para a a componente DC. A corrente que flui para transformador de
corrente e totalmente imune as variações de offset. No caso o transformador de corrente
retira toda a componente DC, logo, não poderá ser vista nenhum sinal DC. Partindo desta
premissa o circuito digital é capaz de aumentar rejeição em modo comum.
Em outras palavras a detecção da componente continua pode ser vista na figura ~\ref{fig:figura3}

\begin{figure}[!ht]
	\centering
	\includegraphics[scale=.8]{fig3.png}
    \caption{Detector da componente DC}
    \label{fig:figura3}
\end{figure}

A função $H$ guarda as informações para a componente DC. Importante notar que os 
altos niveis de saturação não devem ser capazes de modificar a linearidade do 
circuito, por isso o sensor desenvolvido é altamente linear para uma vasta gama 
de valores. O método para o loop fechado apresenta uma resposta mais dinâmica e 
bem mais. A figura ~\ref{fig:figura4} exemplifica todo o processo de aquisição 
da componente DC

\begin{figure}[!ht]
	\centering
	\includegraphics[scale=.8]{fig4.png}
    \caption{Estrutura de controle}
    \label{fig:figura4}
\end{figure}

\section{Resultado das simulações}

\end{document}
 
