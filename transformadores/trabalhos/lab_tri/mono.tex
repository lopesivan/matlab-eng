% Felipe Bandeira da Silva
% 27/09/2013

%\documentclass[a4paper, 10pt]{article}
\documentclass[paper=a4, fontsize=11pt]{article}

%\usepackage[framed,numbered,autolinebreaks,useliterate]{mcode}

\usepackage[brazil]{babel}
\usepackage[utf8]{inputenc}
\usepackage{listings}
\usepackage{color}
\usepackage{amsthm}
\usepackage{graphicx}

\usepackage{schemabloc}
\usetikzlibrary{circuits}

\usepackage{tabularx,ragged2e,booktabs,caption}

\setlength{\parindent}{0pt}
\setlength{\parskip}{18pt}

\title{Laboratório de Transformadores\\Transformação Trifásica com Transformadores Monofásicos}
\author{Felipe Bandeira da Silva\\1020942-X}
%\date{}

\begin{document}


\maketitle

%\newpage

%\begin{abstract}
\textit{Este laboratório tem como objetivo: Exemplificar as devidas ligações que tornam um
transformador monofásico apto a trabalhar em conjunto com outros transformadores monofásico
em uma rede trifásica equilibrada.}
%\end{abstract}

\newpage

\tableofcontents

\newpage

\listoffigures


%%%%%%%%%%%%%%%%%%%%%%%%%%%%%%%%%%%%%%%%%%%%%%%%%%%%%%%%%%%%%%%%%%%%%%%%%%%%%%%%
% fundamentação teórica
\newpage
\section{Fundamentação Teórica}

Um sistema trifásico tem como principal característica a defasagem de $120º$ entre
as três fases. Para facilitar a análise para tal sistema, todo um equacionamento
é desenvolvido tornando conhecido as seguintes relações:

\begin{equation}
    V^{FF} = \sqrt{3} \cdot V^{FN}
\end{equation}

Onde, $V^{FF}$ é conhecida como tensão fase-fase ou tensão de linha, $V^{FN}$ chamada
de tensão fase-neutro ou apenas tensão de fase. Observação: a equação (1) só é válida
para sistema trifásico equilibrado. Um sistema trifásico equilibrado apresenta uma
característica bastante peculiar, a corrente do neutro é igual a zero, qualquer
variação diferente de zero no eletrodo para o neutro, caracteriza um sistema não mais
equilibrado e sim desequilibrado. Um transformador também não é diferente e toda a teoria
estudada para o seu funcionamento em sistemas monofásicos também pode ser utilizada em
sistemas trifásicos. Basta portanto, para isso, considerar o diagrama unifilar da rede.
Para se transformar a tensão de uma fonte trifásica, se requer ou bancada de transformadores
monofásicos, ou, alternativamente  um único transformador trifásico com seis enrolamentos
num núcleo comum de ferro. Necessário apenas o devido cuidado com a fase de cada
enrolamento.
\end{document}

