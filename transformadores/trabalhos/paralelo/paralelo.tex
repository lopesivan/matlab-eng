% Relatório do laboratório 5 de servo
% Felipe Bandeira da Silva
% 27/09/2013

%\documentclass[a4paper, 10pt]{article}
\documentclass[paper=a4, fontsize=11pt]{article}


\usepackage[brazil]{babel}
\usepackage[utf8]{inputenc}
\usepackage{listings}
\usepackage{color}
\usepackage{amsthm}
\usepackage{graphicx}

\usepackage{schemabloc}
\usetikzlibrary{circuits}

\usepackage{tabularx,ragged2e,booktabs,caption}

\setlength{\parindent}{0pt}
\setlength{\parskip}{18pt}

\title{\textsc{Laboratório Transformadores\\Transformadores em Paralelo}}
\author{Felipe Bandeira da Silva\\1020942-X}
%\date{}

\begin{document}


\maketitle

%\newpage

%\begin{abstract}
\textit{Este laboratório tem como objetivo: Aprender como se ligam transformadores
em paralelo. Determinar a eficiência dos transformadores ligados em paralelo.}
%\end{abstract}

\newpage

\tableofcontents

\newpage

%\listoffigures


%%%%%%%%%%%%%%%%%%%%%%%%%%%%%%%%%%%%%%%%%%%%%%%%%%%%%%%%%%%%%%%%%%%%%%%%%%%%%%%%
% fundamentação teórica
\newpage
\section{Fundamentação Teórica}

Os transformadores podem ser ligados em paralelo para proporcionar correntes de carga
maiores que a corrente nominal de cada transformador. Provindo maior confiabilidade
ao sistema elétrico e como consequência maior continuidade de serviço e segurança.
Quando os transformadores monofásicos são ligados em paralelo é necessário levar
em conta as seguintes regras:
\begin{itemize}
    \item Os enrolamentos a serem ligados em paralelo devem ter o mesmo valor nominal
        de tensão de saída.
    \item Os enrolamentos a serem ligados em paralelo devem ter polaridades idênticas.
    \item Relação das resistências equivalentes em ohms igual a relação das reatâncias
        em ohms.
    \item Mesmo barramento de Alta Tensão e o mesmo barramento de Baixa Tensão.
    \item O mesmo deslocamento angular.
    \item A mesma sequência de fases.
\end{itemize}
Recomenda-se que a potência dos trafos interligados em paralelo sejam no máximo 3:1 
para que as relações de reatância de curto circuito sejam a diferença inferior a 10$\%$.
Se estas regras não forem seguidas, podem ser geradas correntes de \textbf{Circulação}
semelhantes a correntes de curto-circuito excessivamente grandes. Com efeito, os 
transformadores, os interruptores e os circuitos associados podem sofrer graves danos
e inclusive explodir, se as correntes de curto-circuito alcançarem certo nível.

\section{Prática, o sistema em paralelo}

Para tanto foi feita a configuração nos módulos EMS, os secundários e 
primários foram ligados em paralelos. O primário de Alta tensão foi energizado
com 120 $V_CA$ e as seguintes medições foram feitas,

\begin{center}
\begin{tabular}{c||c}
    $E_L$ & 0\\
    $I_L$ & 0\\
    $I_1$ & 0\\
    $I_2$ & 0\\
    $P_{entrada}$ & 0\\
    $Q_{entrada}$ & 0\\
\end{tabular}
\end{center}

\subsection{Cálculos}

Potência da carga: $0$
\newline
Eficiência do circuito: $0$
\newline
Perdas do transformador: $0$
\newline
Potência entregue pelo transformador 1: $0$
\newline
Potência entregue pelo transformador 2: $0$

\section{Conclusão}

Após o experimento prático é notado que é possível implementar uma solução bastante
prática, onde existem diversos transformadores de baixa potência quando se pode colocar
este em uma sistema de alta potência ou que requer mais potência.
\end{document}

